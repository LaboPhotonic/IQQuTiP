
\chapter{Installation de QuTiP et commandes usuelles}
\label{sec:QuTiP}

QuTiP (\textbf{Qu}antum \textbf{T}oolbox \textbf{i}n \textbf{P}ython) est un 
logiciel \emph{libre ou open-source} de calculs d'optique quantique avec des 
applications en information quantique. C'est un excellent outil d'appropriation 
et de simulation des concepts fondamentaux de la théorie quantique. Grâce à ce 
logiciel dont la maîtrise est aisée, l'étudiant peut facilement représenter un 
état quantique ou un opérateur, calculer une valeur moyenne, simuler l'évolution 
d'un système, implémenter des algorithmes de l'information quantique.

QuTiP étant rédiger en Python, il est nécessaire, pour utilisant optimale, 
d'avoir quelques notions de base du langage de programmation Python. Pour 
cela nous vous conseillons le cours gratuit \emph{Apprenez à programmer 
en Python} disponible sur le site du zéro \url{
http://uploads.siteduzero.com/pdf/223267-apprenez-a-programmer-en-python.pdf}. 
Il est important de souligner que Python est un langage de programmation 
\textbf{interprété}, c'est-à-dire que les instructions que vous lui envoyez sont 
transcrites en langage machine au fur et à mesure de leur lecture. Les langages 
comme le C / C++ ou le fortran sont appelés \textbf{langages compilés} car, 
avant de pouvoir les exécuter, un logiciel spécialisé se charge de transformer 
le code du programme en langage machine par la \emph{compilation}. À chaque 
modification du code, il faut rappeler une étape de compilation.
 
Nous présentons ici les étapes à suivre pour l'installation de QuTiP sous le 
système d'exploitation libre \textbf{Linux}. Il est à noter que QuTiP fonctionne 
aussi sous les systèmes d'exploitation \textbf{Mac OS X} et \textbf{Windows}. 

\section{Installation de QuTiP pour Ubuntu 12.04 et plus récente}

Pour commencer, il faut déjà un ordinateur sur lequel est installé une variante 
la distribution Linux Ubuntu (Ubuntu, Xubuntu, Kubuntu, Lubuntu, etc.). Se 
connecter à Internet pour télécharger les \emph{paquets} nécessaires\footnote{Il 
faut aussi préciser qu'il y a deux possibilités d'installer QuTiP : manuellement 
et automatiquement. Nous allons présenter seulement l'installation automatique 
pour sa simplicité. Pour l'installation manuel et l'installation sur d'autres 
systèmes d'exploitation veuillez consulter le manuel de QuTiP \emph{QuTiP : The 
Quantum Toolbox in Python release 2.2.0} disponible sur le site 
\url{http://qutip.googlecode.com/files/QuTiP-2.2.0-DOC.pdf}.}

Aller sur le terminal et entrez la commande suivante, pour ajouter le dépôt 
de QuTiP à la \emph{sourcelist}:\\
\begin{tt}
sudo add-apt-repository ppa:jrjohansson/qutip-releases\\
\end{tt}
Mettre ensuite cette liste de dépôt grâce à la commande\\
\begin{tt}
sudo apt-get update\\
\end{tt}
et installer QuTiP avec la commande\\
\begin{tt}
sudo apt-get install python-qutip\\
\end{tt}

Il est recommandé d'installer d'autres paquets afin de compléter l'installation. 
Pour cela dans votre terminal, tapez les commandes suivantes:\\
\begin{tt}
sudo apt-get install texlive-latex-extra\\
sudo apt-get install python-nose
\end{tt}

La dernière étape de la procédure consiste à installer une console Python ou 
interpréteur. A cet effet nous suggérons \texttt{IPython} ou \texttt{bpython} 
qui permet une bonne complétion. Taper sur le terminal\\
\texttt{sudo apt-get install ipython}\\
ou\\
\texttt{sudo apt-get install bpython}

\section{Vérification de l'installation}

Il est possible de vérifier si votre installation de QuTiP s'est bien dérouler. 
Le temps de cette vérification est fonction de la puissance de votre ordinateur. 
Pour vérifier, il faut taper sur terminal les commandes :\\
\begin{tt}
ipython \# ou bpython si vous l'avez installé\\
import qutip.testing as qt\\
qt.run()
\end{tt}

Une fois la vérification faite vous pouvez utiliser QuTiP.

\section{Commandes usuelles}

\subsection{Quantum object class}
Expliquer clairement la notion de Qobj()

\subsection{États et opérateurs}
Donner ici les Qobj prédéfinis pour les états et les opérateurs qui seront 
utilisés dans les divers exercices de ce cours.

\subsection{Les attribues d'une classe Qobj}

Expliquer et donner le tableau.

\subsection{Qobj Math}

Expliquer à travers quelques exemples.

\subsection{Fonction opérant sur une classe Qobj}

En plus des attribues, une classe d'objet quantique à des fonctions qui 
agissent sur les instances Qobj. Les plus usuelles sont:

Donner le tableau et illustrer par quelques exemples.

\section{Manipuler les états et opérateurs}

\subsection{États ket et bra}

Expliquer en prenant des exemples compatibles avec le niveau de compréhension 
ou de connaissances des étudiants.

\subsection{Qubit}

Exemples simples.

\subsection{Valeurs moyennes}

Exemples simples.

\section{Produit tensoriel et trace partielle}

Présenter tenant compte du niveau de connaissances du cours.

\subsection{Produit tensoriel}

Exemples simples.

\subsection{Trace partielle}

Exemples simples.

\section{Sphère de Bloch}


\subsection{Bloch class}

Exemples simples.

\subsection{Bloch3d class}

Exemples simples.

\subsection{Différences entre Bloch and Bloch3d}


\section{Évolution temporelle unitaire}

