\chapter{Installation de QuTiP et commandes usuelles}
\label{sec:QuTiP}

\minitoc

\bigskip


QuTiP, \textbf{Qu}antum \textbf{T}oolbox \textbf{i}n \textbf{P}ython, est un 
logiciel \emph{libre ou open-source} de calculs en optique quantique avec des 
applications en information quantique\footnote{J. R. Johansson, P.D. Nation, and 
F. Nori, \emph{QuTiP 2: A Python framework for the dynamics of open quantum 
systems}, Comp. Phys. Comm. 184, 1234 (2013) ou 
\url{http://arxiv.org/abs/1211.6518}.}. C'est un excellent outil d'appropriation 
et de simulation des concepts fondamentaux de la théorie quantique. Grâce à ce 
logiciel dont la maîtrise est aisée, l'étudiant peut facilement représenter un 
état quantique ou un opérateur, calculer une valeur moyenne, simuler l'évolution 
d'un système, implémenter des algorithmes de l'information quantique.

QuTiP étant rédiger en Python, il est nécessaire, pour utilisant optimale, 
d'avoir quelques notions de base du langage de programmation Python. Pour 
cela nous vous conseillons le cours gratuit \emph{Apprenez à programmer 
en Python} disponible sur le site du zéro \url{
http://uploads.siteduzero.com/pdf/223267-apprenez-a-programmer-en-python.pdf}. 
Il est important de souligner que Python est un langage de programmation 
\textbf{interprété}, c'est-à-dire que les instructions que vous lui envoyez sont 
transcrites en langage machine au fur et à mesure de leur lecture. Les langages 
comme le C/C++ ou le fortran sont appelés \textbf{langages compilés} car, avant 
de pouvoir les exécuter, un logiciel spécialisé se charge de transformer le code 
du programme en langage machine par la \emph{compilation}. À chaque modification 
du code, il faut rappeler une étape de compilation.
 
Nous présentons ici les étapes à suivre pour l'installation de QuTiP sous le 
système d'exploitation libre \textbf{Linux}. Il est à noter que QuTiP fonctionne 
aussi sous les systèmes d'exploitation \textbf{Mac OS X} et \textbf{Windows}. 
La version présentée et utilisée ici est la 2.2.0 (site web: 
\url{http://qutip.googlecode.com}; blog: \url{http://qutip.blogspot.com}).

\section{Installation de QuTiP pour Ubuntu 12.04 et plus récente}

Il faut disposer d'un ordinateur sur lequel est installé une variante la 
distribution Linux Ubuntu (Ubuntu, Xubuntu, Kubuntu, Lubuntu, etc.). Se 
connecter à Internet pour télécharger les \emph{paquets} nécessaires 
en suivant la procédure ci-dessous.\footnote{Il faut préciser qu'on peut 
installer QuTiP aussi bien manuellement qu'automatiquement. Cependant, nous ne 
présentons ici que l'installation automatique du fait de sa simplicité. Pour 
l'installation manuel et l'installation sur d'autres systèmes d'exploitation 
veuillez consulter le manuel de QuTiP \emph{QuTiP : The Quantum Toolbox in 
Python release 2.2.0} disponible sur le site 
\url{http://qutip.googlecode.com/files/QuTiP-2.2.0-DOC.pdf}.}

Aller sur le terminal et entrer les commandes suivantes, pour ajouter le dépôt 
de QuTiP à la \emph{sourcelist}; mettre ensuite cette liste de dépôt à jour; 
et enfin installer QuTiP:\\
\begin{lstlisting}
sudo add-apt-repository ppa:jrjohansson/qutip-releases\\
sudo apt-get update\\
sudo apt-get install python-qutip\\
\end{lstlisting}

Il est recommandé d'installer d'autres paquets, pour utiliser \LaTeX dans les 
figures et effectuer des tests, afin de compléter l'installation. Pour cela 
entrer sur le terminal les commandes suivantes:\\
\begin{lstlisting}
sudo apt-get install texlive-latex-extra\\
sudo apt-get install python-nose
\end{lstlisting}

La dernière étape de la procédure consiste à installer une console Python ou 
interpréteur. A cet effet nous suggérons \texttt{IPython} ou \texttt{bpython} 
qui permet une bonne complétion. Taper sur le terminal\\
\texttt{sudo apt-get install ipython}\\
ou\\
\texttt{sudo apt-get install bpython}

\section{Vérification de l'installation}

Il est possible de vérifier si l'installation de QuTiP s'est bien dérouler. Le 
temps de cette vérification est fonction de la puissance de votre ordinateur. 
Pour vérifier, il faut taper sur terminal les commandes :\\
\texttt{ipython} \\
\begin{lstlisting}
import qutip.testing as qt
qt.run()
\end{lstlisting}
Une fois la vérification faite QuTiP peut être utilisé.


\section{Commandes usuelles}

\subsection{Charger les modules QuTiP}

Avant toute utilisation de QuTiP, il faut importer ses modules avec la commande
\begin{lstlisting}
from qutip import *
\end{lstlisting}
qui met toutes les fonctions et classes de QuTiP, dont le diagramme est donnée 
par la figure \ref{fig:qutip_org}, à disposition pour la suite du programme.

Il est aussi important d'importer la bibliothèque \emph{SciPy} avec la commande
\begin{lstlisting}
from scipy import *
\end{lstlisting}
SciPy est un projet visant à unifier et fédérer un ensemble de bibliothèques 
Python à usage scientifique (fonctions spéciales, interpolation, intégration, 
optimisation, traitement d'images). Scipy utilise les tableaux et matrices du 
module \emph{NumPy}. Lequel module permet d'appliquer des opérations 
simultanément sur l'ensemble d'un tableau permettant d'écrire un code plus 
lisible, plus facile à maintenir et donc plus efficace.

SciPy offre également des possibilités avancées de visualisation grâce au 
module \emph{matplotlib}.

\begin{figure}[htbp]
 \centering
 \includegraphics[scale=.95]{graphics/qutip_org.png}
 % qutip_org.png: 1000x1013 pixel, 130dpi, 19.58x19.83 cm, bb=0 0 555 562
 \caption{Diagramme des classes et fonctions prédéfinies de QuTiP.}
 \label{fig:qutip_org}
\end{figure}

\subsection{Quantum object class}

La classe \texttt{qutip.Qobj} ou \texttt{Qobj()} est une classe de QuTiP qui 
crée et manipule les objets quantiques, sous forme de matrices. Cette classe 
possède 
\begin{itemize}
 \item des \emph{attributs} auxquels on peut accéder pour avoir des 
informations sur l'objet quantique 
\begin{itemize}
\item \texttt{dims=}dimension (de l'espace de Hilbert) de la fonction; 

\item \texttt{shape=}dimension de l'argument de la fonction;
 
\item \texttt{type=}'oper', 'ket', 'bra'; 

\item \texttt{isherm=}'True', 'False' (l'opérateur est-il ou non hermitien?);   
\end{itemize}

\item des \emph{fonctions} que l'on peut appliquer à cet objet quantique  
\begin{itemize}

\item soit pour le modifier ou le transformer (i.e., normaliser, conjugué 
hermitien,\ldots);

\item soit pour acquérir des informations sur l'objet quantique (i.e., valeurs 
propres, vecteurs propres,\ldots).                                  
\end{itemize}

\end{itemize}

\emph{Il est à noter que par convention, les classes d'objet en Python comme} 
\texttt{Qobj()} \emph{diffère d'une fonction par l'utilisation d'une lettre 
majuscule au début.}

\subsection{États et opérateurs}

Avec QuTiP, il y a deux façon de créer des états et des opérateurs :
\begin{itemize}
\item Soit à partir des \texttt{qutip.Qobj} prédéfinis, comme par exemple 
\begin{itemize}
\item \texttt{qutip.states.basis}, \texttt{qutip.states.fock};

\item \texttt{qutip.operators.sigmax}, \texttt{qutip.operators.sigmay}, 
\texttt{qutip.operators.sigmaz}, \texttt{qutip.operators.sigmap} et
\texttt{qutip.operators.sigmam}.
\end{itemize}


\begin{lstlisting}[commentstyle=\scriptsize]
In [1]: basis(2,0) # Vecteur de dimension N=2, de position m=1-1 dans la base
Out[1]: 
Quantum object: dims = [[2], [1]], shape = [2, 1], type = ket
Qobj data =
[[ 1.]
 [ 0.]]

In [2]: basis(2,1) # Vecteur de dimension N=2, de position m=2-1 dans la base
Out[2]: 
Quantum object: dims = [[2], [1]], shape = [2, 1], type = ket
Qobj data =
[[ 0.]
 [ 1.]]

In [3]: fock(2,0) # Vecteur de dimension N=2, de position m=1-1 dans la base
Out[3]: 
Quantum object: dims = [[2], [1]], shape = [2, 1], type = ket
Qobj data =
[[ 1.]
 [ 0.]]

In [4]: fock(2,1) # Vecteur de dimension N=2, de position m=2-1 dans la base
Out[4]: 
Quantum object: dims = [[2], [1]], shape = [2, 1], type = ket
Qobj data =
[[ 0.]
 [ 1.]]

 In [5]: sigmax() # Matrice Sigma-X
Out[5]: 
Quantum object: dims = [[2], [2]], shape = [2, 2], type = oper, isherm = True
Qobj data =
[[ 0.  1.]
 [ 1.  0.]]

In [6]: sigmay() # Matrice Sigma-Y
Out[6]: 
Quantum object: dims = [[2], [2]], shape = [2, 2], type = oper, isherm = True
Qobj data =
[[ 0.+0.j  0.-1.j]
 [ 0.+1.j  0.+0.j]]

In [7]: sigmaz() # Matrice Sigma-Z
Out[7]: 
Quantum object: dims = [[2], [2]], shape = [2, 2], type = oper, isherm = True
Qobj data =
[[ 1.  0.]
 [ 0. -1.]]

In [8]: sigmap() # Matrice Sigma_+
Out[8]: 
Quantum object: dims = [[2], [2]], shape = [2, 2], type = oper, isherm = False
Qobj data =
[[ 0.  1.]
 [ 0.  0.]]

In [9]: sigmam() # Matrice Sigma_-
Out[9]: 
Quantum object: dims = [[2], [2]], shape = [2, 2], type = oper, isherm = False
Qobj data =
[[ 0.  0.]
 [ 1.  0.]]

In [10]: qeye(2) # Matrice identit%é% d'ordre 2
Out[10]: 
Quantum object: dims = [[2], [2]], shape = [2, 2], type = oper, isherm = True
Qobj data =
[[ 1.  0.]
 [ 0.  1.]]
\end{lstlisting}

\item Soit en définissant soi-même les données du \texttt{Qobj()}, comme par 
exemple, pour la matrice carré d'ordre $3\times3$ $A =\begin{pmatrix}
1&1&0\\1&-1&1\\0&1&1\end{pmatrix}$ et les vecteurs ligne 
$\begin{pmatrix}1&2&3&4&5\end{pmatrix}$ et colonne 
$\begin{pmatrix}1\\2\\3\\4\\5\end{pmatrix}$,
\begin{lstlisting}[commentstyle=\scriptsize]
In [11]: Qobj([[1,1,0],[1,-1,1],[0,1,1]]) # Matrice A
Out[11]: 
Quantum object: dims = [[3], [3]], shape = [3, 3], type = oper, isherm = True
Qobj data =
[[ 1.  1.  0.]
 [ 1. -1.  1.]
 [ 0.  1.  1.]]

In [12]: Qobj([1,2,3,4,5]) # Vecteur ligne
Out[12]: 
Quantum object: dims = [[1], [5]], shape = [1, 5], type = bra
Qobj data =
[[ 1.  2.  3.  4.  5.]]

In [13]: Qobj( array([[1],[2],[3],[4],[5]])) # Vecteur colonne
Out[13]: 
Quantum object: dims = [[5], [1]], shape = [5, 1], type = ket
Qobj data =
[[ 1.]
 [ 2.]
 [ 3.]
 [ 4.]
 [ 5.]]
\end{lstlisting}
\end{itemize}

\begin{exercise}
Dans la base $\{\ket{u_1},\ket{u_2},\ket{u_3}\}$ on considère le ket et le bra 
\begin{equation}
\ket{\phi}=\dfrac{1}{2}(\ket{u_1}+\ket{u_2}+\sqrt{2}\ket{u_3}) \qquad 
\bra{\psi}=\dfrac{1}{\sqrt{2}}(\bra{u_1}+\bra{u_3}).
\end{equation}
Sans utiliser des \texttt{qutip.Qobj} prédéfinis, définir $\ket{\phi}$ et 
$\bra{\psi}$.
\end{exercise}

\begin{solution}
Le script peut être le suivant:
\end{solution}
\begin{lstlisting}[commentstyle=\scriptsize]
In [14]: Qobj([[1/2.0],[1/2.0],[1/sqrt(2)]]) # Etat $\ket{\phi}$
Out[14]: 
Quantum object: dims = [[3], [1]], shape = [3, 1], type = ket
Qobj data =
[[ 0.5       ]
 [ 0.5       ]
 [ 0.70710678]]

In [15]: Qobj([1/sqrt(2),1/sqrt(2)]) # Etat $\bra{\psi}$
Out[15]: 
Quantum object: dims = [[1], [2]], shape = [1, 2], type = bra
Qobj data =
[[ 0.70710678  0.70710678]]
\end{lstlisting}

\subsection{Les attributs d'une classe Qobj}

Nous avons dit qu'une classe d'objet \texttt{Qobj()} possède des attributs et 
fonctions. Ces attributs sont présentées sur la Figure \ref{fig:qobj}. La 
Table \ref{tab:class} donne les commandes et descriptions de ces attributs.

\begin{figure}[htbp]
\centering
 \includegraphics{graphics/basics-qobj-box.png}
 % basics-qobj-box.png: 400x272 pixel, 150dpi, 6.77x4.61 cm, bb=0 0 192 131
% \includegraphics[height=6cm]{ClasseQobj.png}
\caption{Attributs d'une classe d'objet \texttt{Qobj()}}
\label{fig:qobj}
\end{figure}

\begin{table}[htpb]
\centering
\begin{tabular}{|l|l|l|} \hline \hline
\textbf{Attributs} &\textbf{Commandes}	&\textbf{Description}\\ \hline \hline
Data	&Q.data	&Matrice représentant Q\\ \hline
Dimensions	&Q.dims	&Permet d'identifier un système composite\\ \hline
Shape	&Q.shape	&Dimension de Q\\ \hline
is Hermitian?	&Q.isherm	&Q est-il hermitien?\\ \hline
Type	&Q.type	&Ket, bra, opérateur ou superopérateur\\ \hline
\end{tabular}
\caption{Description des attributs d'une classe d'objet \texttt{Qobj}.}
\label{tab:class}
\end{table}

\begin{example}
Soit l'opérateur de Pauli $X=\sigma_x$.
\end{example}
\begin{lstlisting}[commentstyle=\scriptsize]
In [16]: X = sigmax() # Matrice Sigma-X

In [17]: X.data # Affiche l'information sur X
Out[17]: 
<2x2 sparse matrix of type '<type 'numpy.complex128'>'
	with 2 stored elements in Compressed Sparse Row format>

In [18]: X.isherm # Indique si X est hermitien ou non
Out[18]: True

In [19]: X.type # Indique la nature de X
Out[19]: 'oper'

In [20]: X.shape # Donne la dimension de X
Out[20]: [2, 2]
\end{lstlisting}

\subsection{Qobj Math}

On peut effectuer sur un objet de classe \texttt{Qobj()}, presque toutes les 
opérations mathématiques qu'on peut effectuer sur des variables ordinaires 
(l'addition, la soustraction, la multiplication, la puissance, etc).\\
\begin{example}
Soit l'opérateur $A =\begin{pmatrix}1&1&0\\1&-1&1\\0&1&1\end{pmatrix}$, et les 
matrices de Pauli $X = \sigma_x$, $Y=\sigma_y$ et $Z = \sigma_z$.
\end{example}
\begin{lstlisting}[commentstyle=\scriptsize]
In [21]: X = sigmax() # Matrcie X

In [22]: Y = sigmay() # Matrice Y

In [23]: Z = sigmaz() # Matrice Z

In [24]: X+Z # Addition des matrices X et Z
Out[24]: 
Quantum object: dims = [[2], [2]], shape = [2, 2], type = oper, isherm = True
Qobj data =
[[ 1.  1.]
 [ 1. -1.]]

In [25]: (1/sqrt(2))*(X+Z) # Produit d'un scalaire et de l'addition de X et Z
Out[25]: 
Quantum object: dims = [[2], [2]], shape = [2, 2], type = oper, isherm = True
Qobj data =
[[ 0.70710678  0.70710678]
 [ 0.70710678 -0.70710678]]
 
In [26]: X*X # $X^2$
Out[26]: 
Quantum object: dims = [[2], [2]], shape = [2, 2], type = oper, isherm = True
Qobj data =
[[ 1.  0.]
 [ 0.  1.]]

In [26]: A=Qobj([[1,1,0],[1,-1,1],[0,1,1]]) # Matrice A

In [27]: A**3 # $A^3$
Out[27]: 
Quantum object: dims = [[3], [3]], shape = [3, 3], type = oper, isherm = True
Qobj data =
[[ 2.  3.  1.]
 [ 3. -3.  3.]
 [ 1.  3.  2.]]
 \end{lstlisting}

\subsection{Fonction opérant sur une classe Qobj}

La Table \ref{tab:fonction} présente les fonctions les usuelles qui agissent 
sur les instances \texttt{Qobj()}.
\begin{table}[!h]
\centering
\begin{tabular}{|l|l|l|} \hline \hline
\textbf{Fonction}	& \textbf{Commande}	& \textbf{Description} \\ 
\hline \hline
Conjugate	& Q.conj()	& Conjugué de Q\\ \hline
Dagger (adjoint)	& Q.dag()	& Conjugué hermitien de Q\\ \hline
Diagonal	& Q.diag()	& Matrice diagonale de Q\\ \hline
Eigenenergies	& Q.eigenenergies()	&  Énergies (valeurs) propres de Q\\ 
\hline
Eigenstates	& Q.eigenstates()	& Valeurs et vecteurs propres de Q\\ 
\hline
Exponential	& Q.expm()	& Matrice exponentielle de Q\\ \hline
Full	& Q.full()	& Donne juste la matrice de Q\\ \hline
Groundstate	& Q.groundstate()	& Énergie et ket de l'état 
fondamentale\\ \hline
Partial Trace	& Q.ptrace(sel)	& Trace partielle choisi par "sel"\\ \hline
Sqrt	& Q.sqrtm()	& Matrice racine carré de Q\\ \hline
Trace	& Q.tr()	&  Trace de Q \\ \hline
Transpose	&  Q.trans()	& Matrice transposé de Q \\ \hline
Unit	& Q.unit()	& Renvoie Q normalisé \\ \hline
\end{tabular}
\caption{Fonction applicable à une classe  d'objet \texttt{Qobj()}}
\label{tab:fonction}
\end{table}

\begin{example}
On considère l'opérateur de Pauli $X$ et l'état $\ket{\psi}=\ket{0}+2\ket{1}$ 
d'un système à deux niveaux.
\end{example}
\begin{lstlisting}[commentstyle=\scriptsize]
In [28]: X = sigmax()

In [29]: X.dag() # Conjugu%é% hermitien de X
Out[29]: 
Quantum object: dims = [[2], [2]], shape = [2, 2], type = oper, isherm = True
Qobj data =
[[ 0.  1.]
 [ 1.  0.]]

In [30]: evals,evecs=X.eigenstates() # Calcul des valeurs propres et 
#vecteurs propres de X

In [31]: evals # Valeurs propres de X
Out[31]: array([-1.,  1.])

In [32]: evecs # Vecteurs propres de X
Out[32]: 
array([ Quantum object: dims = [[2], [1]], shape = [2, 1], type = ket
Qobj data =
[[-0.70710678]
 [ 0.70710678]],
       Quantum object: dims = [[2], [1]], shape = [2, 1], type = ket
Qobj data =
[[ 0.70710678]
 [ 0.70710678]]], dtype=object)

In [33]: X.tr() # Calcul la trace de X
Out[33]: 0.0

In [34]: Psi = basis(2,0)+2*basis(2,1) # D%é%finition de $\ket{\psi}$

In [35]: Psi # Valeurs du $\ket{\psi}$
Out[35]: 
Quantum object: dims = [[2], [1]], shape = [2, 1], type = ket
Qobj data =
[[ 1.]
 [ 2.]]
 
In [36]: Psi.dag() # Valeurs du $\bra{\psi}$
Out[36]: 
Quantum object: dims = [[1], [2]], shape = [1, 2], type = bra
Qobj data =
[[ 1.  2.]]

In [37]: Psi.unit() # Normalisation de $\ket{\psi}$
Out[37]: 
Quantum object: dims = [[2], [1]], shape = [2, 1], type = ket
Qobj data =
[[ 0.4472136 ]
 [ 0.89442719]]
\end{lstlisting}

\begin{exercise}
Dans la base de $\{\ket{0},\ket{1}\}$, utiliser \texttt{qutip.Qobj.unit} pour
normaliser les qubits $\ket{\psi_1}=\ket{0}+3\ket{1}$, 
$\ket{\psi_2}=\ket{0}+\ket{1}$.
\end{exercise}

\subsection{Valeurs moyennes}

La fonction \texttt{qutip.expect}, à travers la commande \texttt{expect(oper, 
state)}, permet de calculer la valeur moyenne d'un opérateur (\texttt{oper}) 
dans un état (\texttt{state}) donné. 

\begin{example}
On considère l'opérateur $X=\sigma_x$, et l'état 
$\ket{\psi}=\frac{1}{\sqrt{2}}(\ket{0}+\ket{1})$. On calcule la valeur moyenne 
de $X$ dans $\ket{\psi}$ par les commandes :
\end{example}
\begin{lstlisting}
X = sigmax()
Psi = (basis(2,0)+basis(2,1))/sqrt(2)
Moyenne = expect(X,Psi)
In [38]: X = sigmax()

In [39]: Psi = (basis(2,0)+basis(2,1))/sqrt(2)

In [40]: Moyenne = expect(X,Psi)

In [41]: Moyenne
Out[41]: 0.9999999999999998
\end{lstlisting}

\subsection{Diagonalisation simultanée}

QuTiP permet de faire la diagonalisation simultanée de matrices hermitiennes 
qui commutent (ECOC) grâce à la commande python
\begin{center}
 \texttt{simdiag([exp\_op\_list], evals=True)}
\end{center}
où \texttt{[exp\_op\_list]} est la liste des opérateurs qui commutent.

\subsection{Sphère de Bloch}
La sphère de Bloch permet de représenter un l'état d'un qubit (un système à 
deux niveaux). La classe \texttt{qutip.Bloch} utilise la bibliothèque
\emph{Matplotlib} pour dessiner la sphère de Bloch à travers la commande 
\texttt{Bloch()}et y ajouter des points, des états ou même des vecteurs avec 
l'une des commandes listée dans la Table (\ref{tab:Bloch}).

\begin{table}[htp]
\centering
\begin{tabular}{|l|p{5.5cm}|p{4.5cm}|}\hline \hline
\textbf{Commande} & \textbf{Variable d'entrée} & \textbf{Description} \\ \hline 
\hline
\texttt{add\_points(pnts,\#meth)} & \texttt{pnts} : liste/tableau des 
points (x,y,z); \texttt{meth}='m' (par défaut 's'): points multicolores & 
Ajoute un ou plusieurs points à la sphère de Bloch \\ \hline
\texttt{add\_states(state,\#kind)} & \texttt{state} : Qobj ou liste de Qobj 
représentant l'état ou l'opérateur densité du système à deux niveaux;  
\texttt{kind}: chaînes de caractères spécifiant si l'état doit 
être représenté comme point ('points') ou vecteur (par défaut)&  Ajoute un ou 
plusieurs états à la sphère de Bloch \\ \hline
\texttt{add\_vectors(vec)} & \texttt{vec} : liste/tableau des points 
(x,y,z) donnant la direction et la longueur du vecteur d'état & 
Ajoute un vecteur ou un ensemble de vecteurs \\ \hline
\texttt{clear()} & & Efface toute donnée sur la sphère de Bloch \\ \hline
\texttt{save(\#format,\#dirc)} & \texttt{format} : format ('png' par défaut) du 
fichier de sortie; \texttt{dirc} : répertoire à utiliser & Sauvegarde la sphère 
 dans un fichier \\ \hline
\texttt{show()} & & Génère la sphère de Bloch\\ \hline
\end{tabular}
\caption{Commandes applicables à la classe \texttt{qutip.Bloch}. Le symbole \# 
signifie que l'argument de la commande est optionnelle.}
\label{tab:Bloch}
\end{table}

Dans l'exemple qui suit, on créé une sphère de Bloch, on y ajoute des 
états et on visualise (voir la figure \ref{fig:bloch_0}).\\
\begin{lstlisting}[commentstyle=\scriptsize]
psi0 = (basis(2,0) + basis(2,1)).unit()
psi1 = (basis(2,0) - 3 * basis(2,1)).unit()
b1=Bloch()
b1.size=[4,4]
b1.font_size=14
b1.add_states([psi0,psi1,basis(2,0),basis(2,1)])
b1.save(dirc='Images') #saving images to Images directory in current working 
#directory
b1.show()
\end{lstlisting}
\begin{figure}[htbp]
\centering
 \includegraphics[scale=.7]{./graphics/bloch_0.png}
 % bloch_0.png: 400x400 pixel, 100dpi, 10.16x10.16 cm, bb=0 0 288 288
 \caption{Sphère de Bloch obtenue avec les états $\ket{0}$, $\ket{1}$, 
$\ket{\psi_0}=\frac{1}{\sqrt{2}}(\ket{0}+\ket{1})$, 
$\ket{\psi_1}=\frac{1}{\sqrt{10}}(\ket{0}-3\ket{1})$.}
 \label{fig:bloch_0}
\end{figure}

\subsection{Évolution temporelle unitaire}

L'évolution d'un système quantique revient à résoudre l'équation d'évolution 
temporelle de Schr\"odinger
\begin{equation}
i\hbar\dfrac{d\ket{\psi}}{dt}=\texttt{H}\ket{\psi}.
\end{equation}
QuTiP utilise la fonction \texttt{qutip.sesolve} pour résoudre l'équation de 
schr\"odinger. La syntaxe est
\begin{center}
 \texttt{sesolve(H, states, tlist, [exp\_op\_list])}
\end{center}
avec
\begin{itemize}
\item \texttt{H}=l'opérateur Hamiltonien;
\item \texttt{states}=l'état initial $\ket{\psi}$;
\item \texttt{tlist}=l'intervalle de temps que l'on définit en utilisant la
commande 
\begin{center}
\texttt{linspace(temps\_initial,temps\_final,pas)};
\end{center}
\item \texttt{[exp\_op\_list]}=la liste des opérateurs dont on évaluera les 
valeurs moyennes pendant \texttt{tlist}.
\end{itemize}

La fonction \texttt{qutip.sesolve}, renvoie à une classe 
d'objet\footnote{ODE=Ordinary Differential Equation, Équation Différentielle 
Ordinaire.} \texttt{qutip.Odedata}. Cette fonction utilise la solution 
(intégrations) d'une équation différentielle ordinaire pour faire évoluer 
l'état $\ket{\psi}$ (\texttt{states}) et évaluer les valeurs moyennes 
des opérateurs (\texttt{exp\_op\_list}) pendant de l'intervalle de temps 
(\texttt{tlist}).

Une classe d'objet générique \texttt{qutip.Odedata} possède plusieurs propriétés 
qui stockent les informations sur la simulation et qui sont regroupées dans la 
Table \ref{tab:Odedata}

\begin{table}[htp]
\centering
\begin{tabular}{|l|p{7cm}|l|} \hline \hline
\textbf{Attributs} & \textbf{Description} & \textbf{Schrödinger}\\ \hline \hline
\texttt{odedata.solver} & Chaîne de caractère précisant la méthode utilisée & 
\texttt{sesolver}\\ \hline
\texttt{odedata.times} & Intervalle de temps pour lequel on a résolu 
l'équation d'évolution & \texttt{tlist}\\ \hline
\texttt{odedata.states} & liste des états calculés pendant \texttt{tlist} & 
\texttt{states} \\ \hline
\texttt{odedata.expect} & Liste des valeurs moyennes calculées & 
\texttt{exp\_op\_list} \\ \hline
\texttt{odedata.num\_expect} & Nombre des valeurs moyennes calculées & $n$ \\ 
\hline
\end{tabular}
\caption{Commandes applicables à la classe \texttt{qutip.Odedata}.}
\label{tab:Odedata}
\end{table}

Pour visualiser l'évolution du système, on utilise la commande 
\begin{center}
\texttt{plot(odedata.times,odedata.expect)}.
\end{center}

\begin{example}
On considère un système à deux niveaux dont l'état initial est 
$\ket{\psi_0}=\ket{0}$ a pour le Hamiltonien
\begin{equation}
\mathtt{H} = \frac{2\pi}{10}\sigma_x.
\end{equation}
La simulation de l'évolution de ce système pendant l'intervalle de temps 
$t\in[0,10]$, pour les valeurs moyennes $\av{\sigma_z}_{\ket{\psi(t)}}$ et 
$\av{P_0}_{\ket{\psi(t)}}$, $P_0=\ket{0}\bra{0}$, peut se fait avec le 
script suivant :
\end{example}
\begin{lstlisting}[commentstyle=\scriptsize]
# coding: utf-8 
#
from qutip import *
from pylab import *

Psi0 = basis(2,0) # vecteur d'%é%tat $\ket{\psi_0}$
H = 2*pi*0.1*sigmax() # Hamiltonien
P0 = basis(2,0)*basis(2,0).dag() # Projecteur $P_0=\ket{0}\bra{0}$
tlist = linspace(0,10,100) # Intervalle de 1 %à% 10 par pas de 10/100
data = sesolve(H,Psi0,tlist,[sigmaz(),P0]) # Evolution du syst%è%me

#Tracer de l'%é%volution

plot(tlist,data.expect[0],tlist,data.expect[1])
xlabel('Time')
ylabel('Valeurs moyennes')
title('Evolution dun systeme')
legend(("Moyenne de Sigma-Z", "Moyenne de P0"))
show()
\end{lstlisting}
Le tracer de cette évolution est donnée par la Figure \ref{fig:EvolSyst1}.

\begin{figure}[htbp]
\centering
 \includegraphics[scale=.6]{./graphics/EvolSyst1.png}
 % EvolSyst1.png: 812x612 pixel, 100dpi, 20.62x15.54 cm, bb=0 0 585 441
 \caption{Evolution dans l'intervalle de temps $t\in[0,10]$ du système 
initialement dans l'état $\ket{0}$ de Hamiltonien $\mathtt{H} = 
\frac{2\pi}{10}\sigma_x$ pour les valeurs moyennes 
$\av{\sigma_z}_{\ket{\psi(t)}}$ et $\av{P_0}_{\ket{\psi(t)}}$.}
 \label{fig:EvolSyst1}
\end{figure}

\subsection{Produit tensoriel} 
\label{sec:tensor}

Lorsqu'on travail sur des systèmes composites (composés de plusieurs sous 
systèmes), on utilise le produit tensoriel entre les états/opérateurs de chaque 
sous-système pour avoir l'état/opérateur du système global étudié.

QuTiP utilise la fonction \texttt{qutip.tensor.tensor} pour effectuer le 
produit tensoriel entre vecteurs d'état ou opérateurs. La syntaxe est 
\texttt{tensor(op1, op2,..., opn)} ou \texttt{tensor([op1, op2,..., opn])}, 
avec op1, op2, ..., opn les opérateurs des sous-systèmes dans le cas 
de l'opérateur global, ou les états des sous-systèmes dans le cas de l'état 
global du système.

On considère par exemple un système composé de trois électrons de spin 1/2. 
Chaque sous-systèmes dispose d'une base $\{\ket{0},\ket{1}\}$. Le système se 
trouve dans l'état initial $\ket{\psi_0}=\ket{000}$. On suppose que le 
Hamiltonien d'évolution du système est 
\begin{equation}
\mathbb{H} = 
\sigma_x\otimes\sigma_y\otimes\mathbb{I}+\sigma_z\otimes\mathbb{I}\otimes W
\end{equation}
où $\sigma_x,\sigma_y,\sigma_z$ sont les trois opérateurs de Pauli et 
$W=\frac{1}{\sqrt{2}}(\sigma_x+\sigma_z)$. La base du système global est 
$\{\ket{000},\ket{001},\ket{010},\ket{011},\ket{100},\ket{101},\ket{110},\ket{
111}\}$.

Le script évaluant l'action de \texttt{H} sur $\ket{\psi_0}$ peut être le 
suivant:\\
\begin{lstlisting}[commentstyle=\scriptsize]
# coding: utf-8 
#
from qutip import *

ket0 = basis(2,0)
ket1 = basis(2,1)
Psi0 = tensor(ket0,ket0,ket0) # Calcul l'%é%tat $\ket{\psi_0}$

X = sigmax()
Y = sigmay()
Z = sigmaz()
W = (X+Z)/sqrt(2)# Matrice de $W$
I = qeye(2)# Matrice identit%é% d'ordre 2
H = tensor(X,Y,I)+tensor(Z,I,W)# Calcul de $H$

Psi = H*Psi0 # Action de $H$ sur $\ket{\psi_0}$

print (H)
print (Psi)
\end{lstlisting}
On obtient en sortie,
\begin{lstlisting}
Quantum object: dims = [[2, 2, 2], [2, 2, 2]], shape = [8, 8], type = oper, 
isherm = True
Qobj data =
[[ 0.70710678+0.j  0.70710678+0.j  0.00000000+0.j  0.00000000+0.j
   0.00000000+0.j  0.00000000+0.j  0.00000000-1.j  0.00000000+0.j]
 [ 0.70710678+0.j -0.70710678+0.j  0.00000000+0.j  0.00000000+0.j
   0.00000000+0.j  0.00000000+0.j  0.00000000+0.j  0.00000000-1.j]
 [ 0.00000000+0.j  0.00000000+0.j  0.70710678+0.j  0.70710678+0.j
   0.00000000+1.j  0.00000000+0.j  0.00000000+0.j  0.00000000+0.j]
 [ 0.00000000+0.j  0.00000000+0.j  0.70710678+0.j -0.70710678+0.j
   0.00000000+0.j  0.00000000+1.j  0.00000000+0.j  0.00000000+0.j]
 [ 0.00000000+0.j  0.00000000+0.j  0.00000000-1.j  0.00000000+0.j
  -0.70710678+0.j -0.70710678+0.j  0.00000000+0.j  0.00000000+0.j]
 [ 0.00000000+0.j  0.00000000+0.j  0.00000000+0.j  0.00000000-1.j
  -0.70710678+0.j  0.70710678+0.j  0.00000000+0.j  0.00000000+0.j]
 [ 0.00000000+1.j  0.00000000+0.j  0.00000000+0.j  0.00000000+0.j
   0.00000000+0.j  0.00000000+0.j -0.70710678+0.j -0.70710678+0.j]
 [ 0.00000000+0.j  0.00000000+1.j  0.00000000+0.j  0.00000000+0.j
   0.00000000+0.j  0.00000000+0.j -0.70710678+0.j  0.70710678+0.j]]
Quantum object: dims = [[2, 2, 2], [1, 1, 1]], shape = [8, 1], type = ket
Qobj data =
[[ 0.70710678+0.j]
 [ 0.70710678+0.j]
 [ 0.00000000+0.j]
 [ 0.00000000+0.j]
 [ 0.00000000+0.j]
 [ 0.00000000+0.j]
 [ 0.00000000+1.j]
 [ 0.00000000+0.j]]
\end{lstlisting}

\subsection{Trace partielle}

La trace partielle peut être comprise comme l'opération inverse du produit 
tensoriel. Elle permet d'extraire l'état/opérateur d'un ou des plusieurs 
sous-système contenu dans l'état/opérateur du système global.

QuTiP utilise la classe \texttt{ qutip.Qobj.ptrace}, pour calculer la trace 
partielle à travers la commande \texttt{ptrace(Op,sel)}, où \texttt{Op} est 
l'état/opérateur du système global et \texttt{sel} est le/les numéro(s) du/des 
sous-système(s) à extraire. La commande \texttt{ptrace()} renvoie le résultat 
sous la forme d'un opérateur projecteur lorsqu'il s'agit des états 
($\ket{\psi}\bra{\psi}$ lorsque l'état voulu est $\ket{\psi}$).

On considère le système décrit à la section (\ref{sec:tensor}). On extrait 
l'état des sous-systèmes 1, 2 et 3 de l'état $\ket{\psi}=H\ket{\psi_0}$ par le 
script suivant:\\
\begin{lstlisting}
Psi1=Psi.ptrace(0) # ou Psi1 = ptrace(Psi,0)
Psi2=Psi.ptrace(1) # ou Psi2 = ptrace(Psi,1)
Psi3=Psi.ptrace(2) # ou Psi3 = ptrace(Psi,2)

print (Psi1)
print (Psi2)
print (Psi3)
\end{lstlisting}
On obtient à la sortie,
\begin{lstlisting}
Quantum object: dims = [[2], [2]], shape = [2, 2], type = oper, isherm = True
Qobj data =
[[ 1.  0.]
 [ 0.  1.]]
Quantum object: dims = [[2], [2]], shape = [2, 2], type = oper, isherm = True
Qobj data =
[[ 1.  0.]
 [ 0.  1.]]
Quantum object: dims = [[2], [2]], shape = [2, 2], type = oper, isherm = True
Qobj data =
[[ 1.5  0.5]
 [ 0.5  0.5]]
\end{lstlisting}

\texttt{Psin} que nous venons de calculer est en fait égale à 
$P_n=\ket{\psi_n}\bra{\psi_n}$.

\subsection{Sauvegarder les objets QuTiP et les données}

Il arrive très souvent qu'on ait besoin de réutiliser les résultats de nos 
calculs pour une session ultérieur ou d'exporter ces données pour en exploiter 
avec d'autres logiciels que QuTiP. Pour cela il nous faut pouvoir sauvegarder 
ces données.

\begin{enumerate}
\item Le contenu des objets quantiques  (qutip.Qobj, qutip.Odedata, etc.) peut 
être stocké dans un fichier et lu ultérieurement respectivement avec les 
fonctions \texttt{qutip.fileio.qsave} et \texttt{qutip.fileio.qload} à travers 
les syntaxes 

\begin{center}
\texttt{qsave(data, name='filename')},
\end{center}
qui sauvegarde le \emph{data} dans fichier '\emph{filename.qu}' dans le 
répertoire courant;
\begin{center}
qload('filename'),
\end{center}
qui charge le contenu du fichier '\emph{filename.qu}' du répertoire courant.

\item Lorsque les données exportées dans un programme autre que QuTiP ou 
python, il est préférable de les stocker sous forme d'un fichier texte de 
type CSV (comma-separated values) pour d'autres calculs ou TSV (tab-separated 
values) pour une visualisation (tracer de figures). On utilise alors, pour 
stocker et lire ou charger les données, les fonctions 
\texttt{qutip.file\_data\_store} et \texttt{qutip.file\_data\_read} à travers 
les syntaxes
\begin{center}
\texttt{file\_data\_store('filename.dat', data, numtype="complex", 
numformat="decimal", sep=",")},
\end{center}
où '\emph{filename.dat}' est le nom du fichier, \emph{data} est le contenu à 
stocker (un vecteur \textbf{numpy}), \emph{numtype} (optionel) est un drapeau 
indiquant le type valeurs numériques (complexe ou réelle), \texttt{numformat} 
(optionel) est un drapeau spécifiant le format numérique des valeurs 
(\texttt{exp} pour le format 1.0e1 et \texttt{decimal} pour le format 
10.0), et \emph{sep} indiquant le type de séparation (tab, space, comma, 
semicolon, etc.);

\begin{center}
file\_data\_read('filename.dat', sep=",").
\end{center}
\end{enumerate}
