\chapter{Calculs quantiques}
\label{sec:CalQ}
\minitoc

\bigskip

Qu'est-ce qu'un calculateur quantique, ou comme on le dit abusivement, un
ordinateur quantique? Ce chapitre nous permettra d'en acquérir les éléments de
base à travers les notions de circuits quantiques (\textbf{section
\ref{sec:CircuitsQ}}), de portes logiques quantiques universelles
(\textbf{section \ref{sec:PQUniv}}). Pour effectuer des calculs de façon
automatique, il faut des algorithmes. Bien qu'il n'existe pour le moment que
très peu d'algorithmes quantiques, elles mettent toutes en exergue la
supériorité des calculs quantiques par rapport aux calculs classiques. A la
\textbf{section \ref{sec:DeutJ}}, nous présentons l'un des tous premiers de ces
algorithmes, l'algorithme de Deutsch-Jozsa. Mais commençons par revisiter la
notion de calculateur à la \textbf{section \ref{sec:calc}}.

\section{Notion de calculateur}
\label{sec:calc}

Un état de $n$ bits d'un calculateur classique ou \textbf{registre classique}
de taille $n$, ne peut stocker, en instant donné, qu'\textbf{un seul entier}
$i\in[0,2^n-1]$ décrit en notation binaire par%
\begin{equation}
i=i_{n-1}2^{n-1}+\cdots+i_12^1+i_02^0=\sum_{m=0}^{n-1}i_{m}2^{m},
\end{equation}
où $i_{m}\in[0,1]$. Ainsi, $3$ bits physiques peuvent être préparés dans
$2^{3}=8$ configurations différentes, représentant les nombres de $0$ à $7$.
Par exemple, les chaînes binaires%
\begin{subequations}%
\begin{align}
011,\\
111,
\end{align}%
\end{subequations}%
représentent respectivement les nombres $3$ et $7$.

Un calculateur quantique est une collection de $n$ qubits qui représente un
\textbf{registre quantique} de taille $n$. L'état de $n$ qubits d'un
calculateur quantique est%
\begin{equation}%
\label{eq:QSuperp}%
\begin{split}
\ket{\psi} &  =\sum_{i=0}^{2^n-1}c_{i}\ket{i} \\
&  =\sum_{i_{n-1}=0}^1\cdots\sum_{i_1=0}^1\sum_{i_0=0}^1c_{i_{n-1},
\cdots i_1,i_0}\ket{i_{n-1}}\otimes\cdots\otimes\ket{i_1}
\otimes\ket{i_0} \\
&  =\sum_{i_{n-1},\cdots i_1,i_0=0}^1c_{i_{n-1},\cdots i_1,i_0}
\ket{i_{n-1}\cdots i_1i_0}
\end{split}%
\end{equation}%
avec la contrainte (complétude)%
\begin{equation}
\sum_{i=0}^{2^n-1}|c_{i}|^2=1.
\label{eq:Complet2}%
\end{equation}

Ainsi, en vertu du principe de superposition clairement visible dans l'Eq.
(\ref{eq:QSuperp}), un registre quantique de $n$ qubits peut être préparé non
seulement dans l'état $\ket{i} $ de la base de calcul, mais
aussi dans une superposition d'états et donc stocker $2^n$ nombres, qui
augmente exponentiellement avec le nombre de qubits. Par conséquent, le
principe de superposition offre de nouvelles possibilités de calculs comme le
\textbf{parallélisme} qui permet le calcul en parallèle d'un grand nombre
d'opérations.

Un état de $n$ qubits d'un calculateur quantique est un vecteur d'état d'un
espace de Hilbert de $2^n$ dimensions, construit comme produit tensoriel de
$n$ espaces de Hilbert de $2$ dimensions, une pour chaque qubit. En prenant en
compte la relation de complétude (\ref{eq:Complet2}) et le fait que l'état de
tout système quantique n'est définit qu'à un facteur de phase global près sans
signification physique, l'état d'un calculateur est déterminé par $2(2^n-1)$
paramètres indépendants. Par exemple, pour $n=2$, un état générique de $2$
qubits d'un calculateur quantique s'écrit%
\begin{equation}
\begin{split}
\ket{\psi} &  =c_0\ket{0}+c_1\ket{1} +c_2\ket{2} +c_{3}\ket{3} \\
&  =c_{0,0}\ket{0} \ket{0} +c_{0,1}\ket{0} \ket{1} +c_{1,0}\ket{1} \ket{0}
+c_{1,1}\ket{1}\ket{1} \\
&  =c_{00}\ket{00} +c_{01}\ket{01}+c_{10}\ket{10} +c_{11}\ket{11}
\end{split}
\end{equation}

Dans la suite, $i\in\{0,1\}^n$, i.e., $i$ est une chaîne binaire de taille
$n$, implique que $\ket{i} $ appartient à l'espace de Hilbert à $2^n$
dimensions $\mathcal{H}^{\otimes n}$.

Pour effectuer un calcul quantique, il faut effectuer les trois étapes de base
suivantes (voir la figure \ref{fig:calq}):

\begin{itemize}
\item La \textbf{préparation} de $n$ qubits dans l'état initial
$\ket{\psi_{i}(t_0)}$ (input state) au temps $t_0$. Le vecteur d'état
initial est un vecteur de l'espace de Hilbert à $2^n$ dimensions
$\mathcal{H}^{\otimes n}$.

\item L'\textbf{implémentation} de la transformation unitaire désirée ou
souhaitée $U(t,t_0)$ qui agira sur l'état initial en évitant tout interaction
avec l'environnement, $\ket{\psi_{f}(t)} =U(t,t_0)\ket{\psi_{i}(t_0)} $.

\item La \textbf{mesure} à l'instant $t$ sur les $n$ qubits afin d'obtenir
l'état final (ouput state).
\end{itemize}

\begin{figure}[ptbh]
\[
\Qcircuit @C=1em @R=0em {
& \multigate{5}{U(t,t_0)}  & \qw \\
& \ghost{U(t,t_0)} & \qw \\
& \ghost{U(t,t_0)} & \qw \\
& \ghost{U(t,t_0)} & \qw \\
& \ghost{U(t,t_0)} & \qw \\
& \ghost{U(t,t_0)} & \qw \\
 \ustick{t_0} & &\ustick{t}
}
\]
\caption{Schéma de principe d'un calcul quantique: $n$ qubits sont préparés dans
l'état initial $\ket{0} $. Ils subissent une évolution unitaire dans l'espace
$\mathcal{H}^{\otimes n}$ de l'instant $t_0$ à l'instant $t$, décrite par un
opérateur unitaire $U(t,t_0)$ agissant dans $\mathcal{H}^{\otimes n}$. Une
mesure des qubits est effectuée au temps $t$.}%
\label{fig:calq}%
\end{figure}

Il est à noter que l'évolution unitaire $U(t,t_0)$ est \textbf{réversible}:
connaissant le vecteur d'état au temps $t$, on peut remonter à celui au temps
$t_0$ par $U^{-}(t,t_0)=U(t_0,t)$.

\medskip\colorbox[gray]{0.8}{
\parbox[c]{0.9\textwidth}{
\begin{center}
\emph{Un calcul quantique est une évolution quantique.}
\end{center}
}}

\section{Circuits quantiques}
\label{sec:CircuitsQ}

\subsection{Énergie - information - réversibilité}

L'information, malgré son caractère abstrait est porté par un support physique.
Il est donc intéressant de se demander s'\emph{\textbf{il est possible de
calculer sans dissiper de l'énergie}}.

\medskip\colorbox[gray]{0.8}{
\parbox[c]{0.9\textwidth}{
\begin{principe}
\textbf{Landauer.} Chaque fois qu'un bit d'information est effacé, la quantité
d'énergie dissipée dans l'environnement vaut au moins $k_{B}T\ln2$. De façon
équivalente, on dit que l'entropie de l'environnement décroît d'au moins de
$k_{B}\ln2$. $k_{B}$ est la constante de Boltzmann et $T$ est la température
absolue de l'environnement (ordinateur).
\end{principe}
}}\medskip

La valeur $k_{B}T\ln2$ représente donc la limite inférieure théorique de
l'énergie dissipée d'une porte logique irréversible. Cependant, malgré l'énorme
réduction de l'énergie dissipée par porte logique tout au long des décennies
passées du fait des progrès techniques, les ordinateurs actuels dissipent encore
jusqu'à environ $500k_{B}T$ par bit effacé, en raison de la consommation de
l'énergie électrique.

\emph{Puisque le principe de Landauer est lié à l'irréversibilité, il est
légitime de se poser la question de savoir si les opérations logiques
habituelles peuvent être conduites de façon réversible, et donc sans dissipation
de l'énergie, afin de transposer au calcul quantique des algorithmes
classiques\footnote{La plupart des portes logiques classiques, nous le savons,
sont irréversibles car elles correspondent à un passage de $2$ bits à $1$ bit,
et l'état final d'un bit ne permet pas de remonter à l'état initial de deux
bits.}.}

En effet, toute fonction irréversible $f:\{0,1\}^{m}\rightarrow\{0,1\}^n$
peut être transformée en fonction réversible en définissant une fonction%
\begin{equation}
\tilde{f}:\{0,1\}^{m+n}\rightarrow\{0,1\}^{m+n}%
\end{equation}
telle que%
\begin{equation}
\tilde{f}(x,y)=(x,y\oplus^n f(x)),
\end{equation}
où $\oplus^n$ est l'addition modulo $2^n$, $x$ représente $m$ bits lors que
$y$ et $f(x)$ représentent $n$ bits. Puisque $\tilde{f}$ transforme des entrées
distinctes en sorties distinctes, elle est une fonction $(m+n)$-bits inversible.

En effet, puisque $f(x)\oplus f(x)=0, \forall f(x)$,
\begin{equation}
(x,y)\mapsto(x,y\oplus f(x))\mapsto(x,(y\oplus f(x))\oplus f(x))=(x,y).
\end{equation}
Il est donc possible de trouver une porte logique universelle.

Comme la porte \texttt{NAND} et l'opération \texttt{COPY} suffisent à
construire tous les circuits logiques classiques, la transposition au calcul
quantique des algorithmes classiques nécessite le remplacement de l'opérateur
\texttt{NAND} par une opération réversible et de trouver l'équivalent de
l'opérateur \texttt{COPY} sans entrer en conflit avec le théorème de non clonage
quantique. La solution de ce problème est l'objet de la \textbf{Section
\ref{sec:PQUniv}}.

\subsection{Parallélisme quantique}

Dans la notation $\ket{x}$, où le nombre $x$ est un des huit
nombres (en binaire)%
\begin{equation}
\ket{000},\ \ket{001},\ \ket{010},\ \ket{011},\ \ket{100},\ \ket{101} ,\
\ket{110} ,\ \ket{111} ,
\end{equation}
un registre quantique de taille $3$ peut stocker les entiers individuels comme
$3$ ou $7$%
\begin{subequations}%
\label{eq:Qubits1}%
\begin{align}
\ket{0} \otimes\ket{1} \otimes\ket{1}  & \equiv\ket{011}\equiv\ket{3} ,\\
\ket{1} \otimes\ket{1} \otimes\ket{1}  & \equiv\ket{111} \equiv\ket{7} ,
\end{align}%
\end{subequations}%
mais, aussi les stocker simultanément. On parle alors de \textbf{parallélisme
quantique}. En effet, si au lieu de prendre le premier single-qubit dans l'état
$\ket{0} $ ou $\ket{1} $, on le prend plutôt dans l'état superposé
$\frac{1}{\sqrt{2}}(\ket{0}+\ket{1})$, alors%
\begin{equation}
\frac{1}{\sqrt{2}}(\ket{0} +\ket{1})\otimes\ket{1} \otimes\ket{1} =\frac
{1}{\sqrt{2}}(\ket{011}+\ket{111})=\frac{1}{\sqrt{2}}(\ket{3} +\ket{7} ).
\end{equation}
On peut évidement préparer ce registre de taille $3$ dans une superposition
d'état des huit entiers, en mettant chaque single-qubit de (\ref{eq:Qubits1})
dans l'état superposé $\frac{1}{\sqrt{2}}(\ket{0} +\ket{1})$:%
\begin{equation}
\begin{split}
&\frac{1}{\sqrt{2}}(\ket{0}+\ket{1})\otimes\frac{1}{\sqrt{2}}(\ket{0}
+\ket{1})\otimes\frac{1}{\sqrt{2}}(\ket{0}+\ket{1} )\\
&  =\frac{1}{\sqrt{2^{3}}}(\ket{000}+\ket{001}+\ket{010}+\ket{011}
+\ket{100}+\ket{101} +\ket{110}+\ket{111})\\
&  =\frac{1}{\sqrt{2^{3}}}(\ket{0} +\ket{1} +\ket{2} +\ket{3}
+\ket{4} +\ket{5} +\ket{6}+\ket{7})
=\frac{1}{\sqrt{2^{3}}}\sum_{x=0}^{2^{3}-1}\ket{x}
\end{split}
\end{equation}

Ces préparations, et toutes autres manipulations sur les qubits, doivent être
effectués par des opérations unitaires et donc des portes réversibles.
L'opération unitaire la plus générale est une transformation dans l'espace de
Hilbert de dimension $2^n$ des $n$-qubits, $\mathcal{H}^{\otimes n}$, et la
porte logique la plus générale est une matrice $2^n\times2^n$ opérant dans
$\mathcal{H}^{\otimes n}$.

On appellera
\begin{itemize}
\item \textbf{porte logique quantique}, un dispositif qui réalise une
opération unitaire fixe sur un qubit donné, pendant une période de temps donnée;

\item \textbf{réseau ou circuit quantique}, un dispositif constitué de portes
logiques quantiques dont les séquences de calculs sont synchronisées dans le
temps.
\begin{itemize}
\item La \textbf{\emph{taille du circuit}} est le nombre de portes logiques
quantiques qu'il contient.
\item La \textbf{\emph{largeur du circuit}} est le nombre de fils qu'il
contient.
\end{itemize}
\end{itemize}

\medskip\colorbox[gray]{0.8}{
\parbox[c]{0.9\textwidth}{
\emph{Si $f:\{0,1\}^n\mapsto \{0,1\}^{m}$ est calculable
par un circuit réversible de $p$ portes et de largeur $w$ alors il est
calculable \textbf{proprement} par un circuit réversible de $2p + m$ portes et
de largeur $w + m$.}
}}

\subsection{Portes single-qubit}

\emph{Les opérations sur un single-qubit (1-qubit) sont décrites par des
matrices unitaires $2\times2$. Les plus utilisées sont les matrices de Pauli,}
\texttt{X, Y, Z}\emph{, la porte de} \texttt{Walsh-Hadamard} \emph{et la porte}
\texttt{Phase-Shift}.

\subsubsection{Porte de \texttt{Walsh-Hadamard}}

La \textbf{porte de} \texttt{Walsh-Hadamard}, ou de \texttt{Walsh-Hadamard}
simplement, définie par la matrice%
\begin{equation}
\mathtt{W}=\frac{1}{\sqrt{2}}\begin{pmatrix}
1 & 1\\
1 & -1
\end{pmatrix},
\label{eq:MHw}%
\end{equation}
permet de transformer les états de base $\{\ket{0},\ket{1}\}$ en état
superposés,%
\begin{subequations}%
\begin{align}
\mathtt{W}\ket{0}  &  =\frac{1}{\sqrt{2}}(\ket{0}+\ket{1} ),\\
\mathtt{W}\ket{1}  &  =\frac{1}{\sqrt{2}}(\ket{0}-\ket{1} ),
\end{align}%
\end{subequations}%
soit sous forme compacte,
\begin{equation}
\mathtt{W}\ket{k}=\frac{1}{\sqrt{2}}(\ket{0}+(-1)^{k}\ket{1})
=\frac{1}{\sqrt{2}}((-1)^{k}\ket{k} +\ket{1-k}),\ k=\{0,1\},
\end{equation}
ou schématiquement par le diagramme
\[
\Qcircuit @C=1em @R=.7em {
 \lstick{\ket{k}} & \gate{W}&
\rstick{\frac{1}{\sqrt{2}}((-1)^{k}\ket{k}+\ket{1-k})} \qw
}
\]

\medskip\colorbox[gray]{0.8}{
\parbox[c]{0.9\textwidth}{
\emph{Comme $\mathtt{W}^2=\mathbb{I}$, la transformation inverse $\mathtt{W}
^{-1}=\mathtt{W}$. La forme de la matrice (\ref{eq:MHw}) montre que $\mathtt{W}$
est hermitien.}
}}
\medskip

Le diagramme ci-dessous représente un circuit quantique de taille $3$ qui
affecte la transformation de \texttt{Walsh-Hadamard} à $3$ single-qubits,
$\mathtt{W}^{\otimes3}\ket{0}\ket{0}\ket{0}=\mathtt{W}\ket{0}\otimes\mathtt{W}
\ket{0}\otimes\mathtt{W}\ket{0}$:

\hspace{6em}\Qcircuit @C=1em @R=.7em {
\lstick{\ket{0}}  &  \gate{W} & \rstick{\frac{1}{\sqrt{2}}(\ket{0} +\ket{1})}
\qw & \\
\lstick{\ket{0}}  &  \gate{W} & \rstick{\frac{1}{\sqrt{2}}(\ket{0} +\ket{1})
 \mapsto\frac{1}{\sqrt{2^{3}}}(\ket{0}
+\ket{1}+\ket{2}+\ket{3}+\ket{4}+\ket{5}+\ket{6}+\ket{7})} \\
\lstick{\ket{0}}  &  \gate{W} & \rstick{\frac{1}{\sqrt{2}}(\ket{0} +\ket{1})}
\qw &\\
}
\medskip

Le résultat (\emph{output}) est une superposition de tous les huit entiers, de
$0$ à $7$. Si les $3$ single-qubits sont initialement dans un autre état que
$\ket{000}$, le résultat de leur transformation par \texttt{Walsh-Hadamard} est
une superposition de tous les huit entiers, de $0$ à $7$, mais avec la moitié
signée positivement. Par exemple,
\begin{equation}
\mathtt{W}^{\otimes3}\ket{101}=\frac{1}{2^{3/2}}(\ket{0}-\ket{1}+\ket{2}
-\ket{3} +\ket{4} -\ket{5} +\ket{6} -\ket{7})
\end{equation}

En général, si initialement on a un registre de taille $n$ dans un état
$y\in\{0,1\}^n$, alors
\begin{equation}
\mathtt{W}^{\otimes n}\ket{y}=\frac{1}{\sqrt{2^n}}\sum_{x=0}^{2^n-1}(-1)^{yx}
\ket{x}=\frac{1}{\sqrt{2^n}}\sum_{x=0}^{2^n-1}e^{i\pi yx}\ket{x},
\end{equation}
où le produit de $y=(y_0y_{n-2}\cdots y_1y_0)$ et de $x=(x_{n-1}x_{n-2}\cdots
x_1x_0)$ est fait bit par bit,
\begin{equation}
yx=(y_0x_{n-1}+y_{n-2}x_{n-2}+\cdots+y_1x_1+y_0x_0).
\end{equation}

Si l'on prend $\ket{0_{m}}$ comme état initial du registre de résultats, alors%
\begin{equation}
U\ket{x\otimes0_{m}} =\ket{x\otimes f(x)}.
\end{equation}
Si on applique $\mathtt{W}$ sur le registre de données dans l'état
$\ket{0_{n}}$ avant
$U$, le vecteur d'état dans l'état final sera par linéarité%
\begin{equation}
\ket{\psi_{fin}}=U\ket{(\mathtt{W}0_{n})\otimes0_{m}}=\frac{1}{\sqrt{2^n}}
\sum_{x=0}^{2^n-1}\ket{x\otimes f(x)} .
\end{equation}
Ce vecteur contient en principe $2^n$ valeurs de la fonction $f(x)$ obtenu en
une seule action de $U$. Par exemple, si $n=100$, il contient $\sim10^{30}$
valeurs de $f(x)$: c'est le miracle du \textbf{\emph{parallélisme quantique}}.
Mais une mesure sur $\ket{\psi_{fin}}$ ne donnera qu'une et une seule de ces
valeurs. Cependant, on peut extraire des informations utiles sur des relations
entre valeurs de $f(x)$ pour un ensemble de valeurs de $x$ différentes, mais
bien sûr au prix de la perte de ces valeurs individuelles, alors qu'un
ordinateur classique devrait évaluer $f(x)$ $2^n$ fois pour obtenir la même
information. C'est là l'origine de l'accélération exponentielle d'un calcul
quantique pour la résolution de certains problèmes.

\begin{remark}
 Les rotations de la sphère de Bloch autour d'un axe arbitraire
$\boldsymbol{\hat{u}}$, que nous avons étudiée à la \textbf{section
\ref{sec:SpinArbitr1}} sont une classe importante de transformations unitaires:
\begin{equation}
R_{\boldsymbol{\hat{u}}}(\delta)=e^{-i\frac{\delta}{2}(\boldsymbol{\hat{u}}
\cdot\sigma)}=\mathbb{I}\cos\frac{\delta}{2}-(\boldsymbol{\hat{u}}
\cdot\sigma)i\sin\frac{\delta}{2}.
\end{equation}

On note que la porte de \texttt{Walsh-Hadamard} est une opération de rotation
d'angle $\delta=\pi$ autour de l'axe $\boldsymbol{\hat{u}}^{\prime}
=(\frac{1}{\sqrt{2}},0,\frac{1}{\sqrt{2}})$. En effet,%
\begin{equation}
\mathtt{W}=\frac{1}{\sqrt{2}}(\sigma_{x}+\sigma_{z})
=-iR_{\boldsymbol{u}^{\prime}}(\pi).
\end{equation}
\end{remark}

\begin{remark}
 L'application de la porte $\mathtt{W}$ à un 1-qubit arbitraire
$\ket{\psi}=\alpha\ket{0}+\beta\ket{1}$ est un exemple de l'\textbf{interférence
quantique} qui se manifeste mathématiquement par l'addition des amplitudes de
probabilités. En effet,
\begin{equation}
\mathtt{W}\ket{\psi}=\frac{\alpha+\beta}{2}\ket{0}+\frac{\alpha-\beta}{2}\ket{1}
.
\end{equation}

\begin{itemize}
 \item La probabilité d'obtenir $\ket{0}$ après une mesure a augmentée par
\textbf{interférence constructive}:
\begin{equation}
 \alpha\rightarrow\frac{\alpha+\beta}{2}
\end{equation}
\item La probabilité d'obtenir $\ket{1}$ après une mesure a diminuée par
\textbf{interférence destructive}:
\begin{equation}
 \alpha\rightarrow\frac{\alpha-\beta}{2}
\end{equation}
\end{itemize}
Si par exemple $\ket{\psi}=\frac{1}{\sqrt{2}}(\ket{0}+(-1)^x\ket{1})$,
$x\in\{0,1\}$,
\begin{equation}
\ket{\psi}=\frac{1+(-1)^x}{2}\ket{0}+\frac{1+(-1)^x}{2}\ket{1}.
\end{equation}
Pour $x=0$, on a
\begin{equation}
 \mathtt{W}\left(\frac{1}{\sqrt{2}}(\ket{0}+\ket{1})\right)=\ket{0}.
\end{equation}
Ainsi, par interférence constructive, la mesure de $\ket{0}$ est certaine.
Alors que par interférence destructive, on a aucune chance de trouver $\ket{1}$
la mesure.
\end{remark}

\begin{exercise}
 En utilisant $\mathtt{W}=\frac{1}{\sqrt{2}}(\mathtt{X}+\mathtt{Z})$ et les
propriétés des matrices de Pauli, montrer que  $\mathtt{W}^2=\mathbb{I}$,
$\mathtt{W}\mathtt{X}\mathtt{W}=\mathtt{Z}$ et
$\mathtt{W}\mathtt{Z}\mathtt{W}=\mathtt{X}$.
\end{exercise}


\subsubsection{Porte \texttt{Phase-Shift}}

La \textbf{porte \texttt{Phase-Shift}} définie par la matrice%
\begin{equation}
R_{z}(\delta)=\begin{pmatrix}
1 & 0\\
0 & e^{i\delta}
\end{pmatrix},
\end{equation}
et symboliquement représentée par le diagramme de la figure
\ref{fig:PhaseShift}, laisse inchangée l'état de base $\ket{0} $ et change la
phase globale de l'état $\ket{1} $, $e^{i\delta}\ket{1}$.
\begin{figure}[ptbh]
\begin{minipage}[c]{.48\linewidth}
\[
\Qcircuit @C=1em @R=.4em {
\lstick{\ket{k}} & \qw & \gate{R_z(\delta)} \qw & \qw &
\rstick{e^{ik\delta}\ket{k}} \qw
}
\]
\caption{Représentation schématique de l'action de la porte
\texttt{Phase-Shift} sur un single-qubit dans l'état $\ket{k}, \ k=\{0,1\}$.}%
\label{fig:PhaseShift}%
\end{minipage}
\hfill\begin{minipage}[c]{.48\linewidth}
\[
\Qcircuit @C=1em @R=.2em {
\lstick{\ket{0}} & \gate{W} & \gate{R_z(\theta)}\qw & \gate{W}&
\gate{R_z(\frac{\pi}{2}+\varphi)} \qw & \rstick{\ket{\psi}} \qw
}
\]
\caption{Construction du single-qubit générique
$\ket{\psi}=\cos\frac{\theta}{2}\ket{0}+e^{i\varphi}\sin\frac{\theta}{2}\ket{1}$
à partir single-qubit $\ket{0} $.}
\label{fig:SQbitGener}%
\end{minipage}
\end{figure}

La porte de \texttt{Walsh-Hadamard} et la porte \texttt{Phase-Shift} peuvent
être combinées pour construire le circuit de taille $4$ de la figure
\ref{fig:SQbitGener}, qui génère, au facteur de phase globale
$e^{i\frac{\theta}{2}}$ près, le single-qubit générique
\begin{equation}
\ket{\psi}=\cos\frac{\theta}{2}\ket{0}+e^{i\varphi}\sin\frac{\theta}{2}\ket{1} .
\label{eq:SQbitGener}
\end{equation}

Ce circuit quantique s'écrit vectoriellement sous la forme\footnote{Le diagramme
se lit de gauche à droite alors que le produit d'opérateurs se lit de droite à
gauche.}
\begin{equation}
R_{z}(\frac{\pi}{2}+\varphi)\mathtt{W}R_{z}(\theta)\mathtt{W}\ket{0}
=e^{i\frac{\theta}{2}}(\cos\frac{\theta}{2}\ket{0}
+e^{i\varphi}\sin\frac{\theta}{2}\ket{1}).
\end{equation}

\medskip\colorbox[gray]{0.8}{
\parbox[c]{0.9\textwidth}{
\emph{Conséquemment, la porte de \texttt{Walsh-Hadamard} et la porte
\texttt{Phase-Shift} suffisent pour construire \textbf{toute} opération
unitaire sur un single-qubit.}
}}\medskip

Et d'un manière générale, ces deux portes peuvent être utilisées pour
transformer l'état initial $\ket{0_1} \ket{0_2} \ldots\ket{0_{n}}$ d'un
registre de $n$ qubits en n'importe quel état de type $\ket{\psi_1}
\ket{\psi_2}\ldots\ket{\psi_{n}} $, où $\ket{\psi_{i}} $ est un état superposé
arbitraire de $\ket{0}$ et $\ket{1}$. Ce sont ces états $n$-qubits qu'on appelle
\textbf{états produit tensoriel ou état séparables}.

Le tableau (\ref{tab:PorteUnit}) présente les portes unitaires single-qubit les
plus usuelles.
\begin{table}[htbp]
\centering
\begin{tabular}
[c]{|l|p{6cm}|l|}\hline
\rowcolor[gray]{.8}\textbf{Nom} & \textbf{Diagramme} & \textbf{Matrice dans}
$\{\ket{0},\ket{1}\}$\\\hline
\texttt{Walsh-Hadamard} $\mathtt{W}$ & \hspace*{1em} \mbox{\Qcircuit @C=1em
@R=.7em {
\lstick{\ket{k}} & \gate{W} &
\rstick{\frac{1}{\sqrt{2}}((-1)^{k}\ket{k}+\ket{1-k})} \qw
}} & \multicolumn{1}{|c|}{$\frac{1}{\sqrt{2}}\begin{pmatrix}1 & 1\\1 & -1
\end{pmatrix}$}\\\hline
\texttt{Pauli X} & \hspace*{1em} \mbox{\Qcircuit @C=1em @R=.7em {
\lstick{\ket{k}} & \gate{X} & \rstick{\ket{1-k}}\qw
}} & \multicolumn{1}{|c|}{$\begin{pmatrix}0 & 1\\1 & 0
\end{pmatrix}$}\\\hline
\texttt{Pauli Y} & \hspace*{1em}  \mbox{\Qcircuit @C=1em @R=.7em {
\lstick{\ket{k}} & \gate{Y} & \rstick{i(-1)^{k}\ket{1-k}}\qw
}}  & \multicolumn{1}{|c|}{$\begin{pmatrix}0 & -i\\i & 0
\end{pmatrix}$}\\\hline
\texttt{Pauli Z} & \hspace*{1em}  \mbox{\Qcircuit @C=1em @R=.7em {
\lstick{\ket{k}} & \gate{Z} & \rstick{(-1)^{k}\ket{k}}\qw
}}  & \multicolumn{1}{|c|}{$\begin{pmatrix}1 & 0\\0 & -1
\end{pmatrix}$}\\\hline
\texttt{Phase} & \hspace*{1em}  \mbox{\Qcircuit @C=1em @R=.7em {
\lstick{\ket{k}} & \gate{S} & \rstick{(i)^{k}\ket{k}}\qw
}}  & \multicolumn{1}{|c|}{$\begin{pmatrix}1 & 0\\0 & i
\end{pmatrix}$}\\\hline
\texttt{Phase-Shift} & \hspace*{1em}  \mbox{\Qcircuit @C=1em @R=.7em {
\lstick{\ket{k}} & \gate{R_z(\delta)} & \rstick{e^{ik\delta}\ket{k}}\qw
}}  & \multicolumn{1}{|c|}{$\begin{pmatrix}1 & 0\\0 & e^{ik\delta}
\end{pmatrix}$}\\\hline
\end{tabular}
\caption{Représentation symbolique et matricielle des portes unitaires les plus
usuelles.}
\label{tab:PorteUnit}%
\end{table}%

\begin{exercise}
On considère les états $\ket{0} $ et $\ket{1}$ orthonormés, élément de l'espace
de Hilbert $\mathcal{H}$.

\begin{enumerate}
\item On appelle opérateur de \texttt{Hubbard}, les opérateurs $X^{ik}$ $(
i,k=1,2)$ dont les matrices sont carrées et ont un élément de matrice unité à
l'intersection de la $i$-ième ligne et de la $k$-ième colonne. Les autres
éléments de matrices sont nuls. En notation de Dirac,
\begin{equation}
X^{ik}=\ket{i-1}\bra{k-1}.
\end{equation}

\begin{enumerate}
\item Écrire en notation de Dirac les quatre opérateurs $X^{ik}$ et donner leur
forme matricielle.

\item Évaluer $X^{12}\ket{1} $ et $X^{21}\ket{0}$ et conclure.

\item Donner l'expression générale de la multiplication de deux opérateurs
$X^{ik}$, $X^{ik}X^{mn}$.
\end{enumerate}

\item On définit l'opérateur porte logique \texttt{NOT} par%
\begin{equation}
\mathtt{NOT}=X^{12}+X^{21}.
\end{equation}

\begin{enumerate}
\item Exprimer \texttt{NOT} en notation de Dirac et sous forme matricielle.

\item Vérifier que \texttt{NOT} est un opérateur hermitien et unitaire.

\item Évaluer \texttt{NOT}$\ket{1}$ et \texttt{NOT}$\ket{0} $ et conclure.

\item Soit un système représenté par le qubit%
\begin{equation}
\ket{\psi}=\alpha\ket{0} +\beta\ket{1},\text{ avec }|\alpha|^2+|\beta|^2=1.
\end{equation}
Que représente $|\alpha|^2$ et $|\beta|^2$ avant et après l'action de
l'opérateur porte logique \texttt{NOT}?
\end{enumerate}
\end{enumerate}
\end{exercise}

\subsection{Portes de contrôle et génération de l'intrication}
\label{sec:CU}

En général, un registre de taille $n>1$ peut être préparé dans des \textbf{états
intriqués ou non séparables}. On rappelle que, par exemple pour $n=2$, l'état
\begin{equation}
\alpha\ket{00} +\beta\ket{01} =\ket{0}\otimes(\alpha\ket{0}+\beta\ket{}1),
\end{equation}
est séparable, en $\ket{\psi_1} =\ket{0}$ et $\ket{\psi_2} =\alpha\ket{0}
+\beta\ket{1} $. Par contre, l'état
\begin{equation}
\alpha\ket{00} +\beta\ket{11} \neq \ket{\psi_1}\otimes \ket{\psi_2},
\end{equation}
est intriqué ($\alpha, \beta \neq 0$).

Afin d'intriquer au moins deux qubits, il nous faut étendre notre répertoire de
portes logiques quantiques aux portes logiques 2-qubits qui réalise une
dynamique conditionnelle. Ces portes sont des \textbf{portes de contrôle} $U$
qui traduisent quantiquement \texttt{if ( x ) then y := Ux} par
\begin{equation}
\ket{x}\ket{y}\mapsto\ket{x}U^{x}\ket{y},
\label{eq:CU}
\end{equation}
qui correspond, pour $x,y\in\{0,1\}$, à
\begin{subequations}%
\begin{align}
\ket{0} \ket{0}  &  \rightarrow\ket{0}\ket{0} \\
\ket{0} \ket{1}  &  \rightarrow\ket{0}\ket{1} \\
\ket{1} \ket{0}  &  \rightarrow\ket{1}U\ket{0} \\
\ket{1} \ket{1}  &  \rightarrow\ket{1}U\ket{1}
\end{align}%
\end{subequations}%
Usuellement, on l'appelle porte \texttt{Controlled-U ou CU} et on la représente
sous formes de décomposition spectrale et matricielle, dans la base
$\{\ket{00},\ket{01},\ket{10},\ket{11}\}$, par

\begin{minipage}[c]{.48\linewidth}
\[
\ket{0}\bra0\otimes\mathbb{I}+\ket{1}\bra{1}\otimes U=
\begin{pmatrix}
\mathbb{I} & \mathbb{O}\\
\mathbb{O} & U
\end{pmatrix}
\]
\end{minipage} \hfill\begin{minipage}[c]{.48\linewidth}
\[
\Qcircuit @C=1em @R=1.4em {
\lstick{\ket{x}}  &  \ctrl{1} & \rstick{\ket{x}} \qw \\
\lstick{\ket{y}}  & \gate{U}   & \rstick{U^{x}\ket{y}} \qw \\
& \mbox{Porte \texttt{CU}} &
}
\]
\end{minipage}
où $\mathbb{I},$ $\mathbb{O}$ et $U$ sont des matrices $2\times2$.

Le premier bit $\ket{x}$ agit comme \textbf{contrôle} et sa valeur reste
inchangée à la sortie. Le second bit $\ket{y}$ est appelé \textbf{cible}.  Sur
le diagramme, le contrôle est représenté le point noir.

\medskip\colorbox[gray]{0.8}{
\parbox[c]{0.9\textwidth}{
\emph{Une porte} \texttt{CU} \emph{applique la transformation identité
$\mathbb{I}$ au bit cible lorsque le bit de contrôle est dans l'état $\ket{0}$.
Elle applique la transformation $U$ au bit cible lorsque le bit de contrôle est
dans l'état $\ket{1}$.}
}}\medskip

Puisque pour $x\in\{0,1\}$, $U^{2x}=\mathbb{I}$ et les opérateurs \texttt{CU}
sont unitaires.

Pour une transformation unitaire quelconque $U:\ (x,y)\rightarrow(x,y\oplus
f(x))$, on a%
\begin{equation}
\ket{\psi}=\mathtt{CU}(\alpha\ket{0}+\beta\ket{1})\ket{0} =\alpha\ket{0f(0)}
+\beta\ket{1f(1)},
\label{eq:PhiCU}%
\end{equation}
qui contient \textbf{\emph{à la fois}} l'information sur $f(0)$ et sur $f(1)$.

\subsubsection{Porte \texttt{CNOT}}

La plus populaire des portes \texttt{CU} est la porte \texttt{CNOT} ou
\texttt{CX} qui opère la transformation décrite par %
\begin{equation}
\mathtt{CNOT}=\ket{0}\bra0\otimes\mathbb{I}+\ket{1}\bra{1}\otimes\mathtt{X} =
\begin{pmatrix}
\mathbb{I} & \mathbb{O}\\
\mathbb{O} & \mathtt{X}
\end{pmatrix}
=\begin{pmatrix}
1 & 0 & 0 & 0\\
0 & 1 & 0 & 0\\
0 & 0 & 0 & 1\\
0 & 0 & 1 & 0
\end{pmatrix},
\end{equation}
autrement qui inverse le bit cible $\ket{y}$ lorsque le bit de contrôle
$\ket{x}$ est dans l'état $\l\ket{1}$. On l'a résume par

\begin{minipage}[c]{.41\linewidth}
\[
\texttt{CNOT}\ket{x}\ket{y}=\ket{x}\ket{x\oplus y}, \ x,y\in\{0,1\},
\]
\end{minipage} \hfill\begin{minipage}[c]{.25\linewidth}
\[
\Qcircuit @C=1em @R=1.4em {
\lstick{\ket{x}}  &  \ctrl{1} &  \rstick{\ket{x}} \qw \\
\lstick{\ket{y}}  &  \targ  & \rstick{\ket{x\oplus y}} \qw \\
& \mbox{Porte \texttt{CNOT}} &
}
\]
\end{minipage}\hfill\begin{minipage}[c]{.25\linewidth}
\centering
\begin{tabular}
[c]{|l|l||l|c|}\hline
\rowcolor[gray]{.8}$x$ & $y$ & $x$ & $x\oplus y$\\\hline
0 & 0 & 0 & 0\\\hline
0 & 1 & 0 & 1\\\hline
1 & 0 & 1 & 1\\\hline
1 & 1 & 1 & 0\\\hline
\end{tabular}
\end{minipage}
\medskip

On note sur la table de vérité que lorsque la cible est dans l'état $\ket{1}$,
la porte \texttt{CNOT} devient la porte \texttt{COPY}:
\begin{equation}
\ket{x}\ket{0} \mapsto\ket{x}\ket{x}, \ x\in\{0,1\}.
\end{equation}
Par conséquent,
\begin{equation}
\mathtt{CNOT}(\alpha\ket{0} +\beta\ket{1})\ket{0} =\alpha\ket{00}
+\beta\ket{11},
\end{equation}
qui est non factorisable pour $\alpha,\beta\neq0$. Donc, la porte \texttt{CNOT}
génère des états intriqués.

On peut penser que cette porte pourrait aussi être utilisée pour copier un état
superposé comme $\ket{\psi}=\alpha\ket{0}+\beta\ket{1} $, si bien que
\begin{equation}
\ket{\psi}\ket{0} \mapsto\ket{\psi}\ket{\psi}.
\end{equation}
\emph{\textbf{Ceci n'est pas possible}} en vertu du théorème du non-clonage.

\colorbox[gray]{0.8}{
\parbox[c]{0.9\textwidth}{
\begin{theorem}\label{theo:TransU}
Toute opération unitaire sur $\mathcal{H}^{\otimes n}$ peut se décomposer
en produit d'opérations unitaires single qubit (1-qubit) et de \texttt{CNOT}.
\end{theorem}
}}
\begin{exercise}
Implémenter, en utilisant QuTiP, la porte logique quantique $CX$.
\end{exercise}

\begin{exercise}
The most general separable state of the two qubits can be written, up to an
overall phase, as%
\begin{equation}
\ket{\psi}=a(\ket{0} +b_1e^{i\varphi_1}\ket{1})\otimes(\ket{0}
+b_0e^{i\varphi_0}\ket{1}),
\end{equation}
where $a$ is set to the completure. What conditions should the real
coefficients $b_0$, $b_1$, $\varphi_0$ and $\varphi_1$ satisfy in
order that $\mathtt{CNOT}\ket{\psi}$ be entangled?
\end{exercise}

\begin{remark}
Il est possible de définir une porte \texttt{CNOT} généralisée, dépendante du
fait que
\begin{itemize}
\item le bit de contrôle est le premier ou le second qubit,
\item ou encore que la porte agit trivialement\footnote{L'action de la porte
se réduit à l'identité.} quand le bit de contrôle est $\ket{0}$ ou
$\beta\ket{1}$.
\end{itemize}

Ainsi, les trois autres matrices \texttt{CNOT}, leurs représentations
symboliques et actions sont:%
\begin{center}%
\begin{tabular}
[c]{|p{4.8cm}|p{4.8cm}|p{4.8cm}|}\hline\hline
$B=\begin{pmatrix}
0 & 1 & 0 & 0\\
1 & 0 & 0 & 0\\
0 & 0 & 1 & 0\\
0 & 0 & 0 & 1
\end{pmatrix}$ & $C=\begin{pmatrix}
1 & 0 & 0 & 0\\
0 & 0 & 0 & 1\\
0 & 0 & 1 & 0\\
0 & 1 & 0 & 0
\end{pmatrix}$ & $D=\begin{pmatrix}
0 & 0 & 1 & 0\\
0 & 1 & 0 & 0\\
1 & 0 & 0 & 0\\
0 & 0 & 0 & 1\end{pmatrix}$\\\hline\hline
\[\Qcircuit @C=1em @R=1.7em {
& \ctrlo{1} &  \qw \\
& \targ & \qw
}
\]
&
\[\Qcircuit @C=1em @R=1.7em {
 & \targ & \qw \\
 & \ctrl{-1} & \qw
}\]
&
\[\Qcircuit @C=1em @R=1.7em {
 & \targ & \qw \\
 & \ctrlo{-1} & \qw
}
\]
\\\hline
$B$ inverse le $2^{e}$ qubit lorsque le $1^{er}$ est dans l'état $\ket{0}$ &
\multicolumn{1}{|p{4.8cm}|}{$C$ inverse le $1^{er}$ qubit lorsque le $2^{e}$ est
dans l'état $\ket{1} $} & $D$ inverse le $1^{er}$ qubit lorsque le $2^{e}$ est
dans l'état $\ket{1}$\\\hline
\end{tabular}%
\end{center}%
\end{remark}

\subsubsection{Base de Bell}

\label{sec:BBell}

Comme vu à la \textbf{Section \ref{sec:CU}}, la porte \texttt{CNOT} peut
générer l'intrication, et en particulier les états intriqués de la base de
Bell, définis par \ref{eq:BellBase},%
\begin{subequations}%
\begin{align}
\ket{\Phi^{+}}&  =\frac{1}{\sqrt{2}}(\ket{00} +\ket{11} )\\
\ket{\Phi^{-}}&  =\frac{1}{\sqrt{2}}(\ket{00} -\ket{11} )\\
\ket{\psi^{+}} &  =\frac{1}{\sqrt{2}}(\ket{01} +\ket{10} )\\
\ket{\psi^{-}} &  =\frac{1}{\sqrt{2}}(\ket{01} -\ket{10} )
\end{align}%
\end{subequations}%
peuvent être obtenus de la base de calcul $\{\ket{0} ,\ket{1} \}$ à travers le
circuit
\[
\Qcircuit @C=1em @R=1.4em {
\lstick{\ket{x}} & \gate{W}&  \ctrl{1} &  \multimeasureD{1}{\text{Bell}} \\
\lstick{\ket{y}}  & \qw & \targ  & \ghost{\text{Bell}}
}
\]

Il est facile de vérifier que ce circuit produit les transformations%
\begin{equation}
\ket{00} \rightarrow\ket{\Phi^{+}};\ \ket{10} \rightarrow\ket{\Phi^{-}}
;\ \ket{01} \rightarrow\ket{\psi^{+}};\ \ket{11} \rightarrow\ket{\psi^{-}}.
\end{equation}
On note que cette transformation peut être inversée simplement en exécutant le
circuit de la droite vers la gauche, puisque les portes \texttt{CNOT} et de
\texttt{Walsh-Hadamard} sont inversibles. Par conséquent, tout état de la base
de Bell est transformé en état factorisable. Et il est donc possible, via une
mesure standard dans la base de calcul, d'établir laquelle des quatre états de
base de Bell était présente au début.

\subsubsection{Porte \texttt{Controlled Phase-Shift}}

La deuxième porte \texttt{CU} usuelle est la porte \texttt{Controlled
Phase-Shift} définie, dans la base $\{\ket{00},\ket{01},\ket{10},\ket{11}\}$,
par

\begin{minipage}[c]{.48\linewidth}
\[
\mathtt{CPS}(\delta)=\begin{pmatrix}
\mathbb{I} & \mathbb{O}\\
\mathbb{O} & \mathtt{R}_{z}(\delta)
\end{pmatrix}=\begin{pmatrix}
1 & 0 & 0 & 0\\
0 & 1 & 0 & 0\\
0 & 0 & 1 & 0\\
0 & 0 & 0 & e^{i\delta}
\end{pmatrix}
\]
\end{minipage} \hfill\begin{minipage}[c]{.48\linewidth}
\[
\Qcircuit @C=1em @R=1.em {
\lstick{\ket{x}}  &  \ctrl{1} & \qw & \rstick{\ket{x}}\\
\lstick{\ket{y}}  &  \gate{R_z(\delta)} &  \qw &
\rstick{e^{ixy\delta}\ket{y}} \\
& \mbox{Porte \texttt{CPS}} & \\
}
\]
\end{minipage}
\medskip

Elle applique la phase global $e^{i\delta}$ lorsque le qubit de contrôle
$\ket{y}$ est dans l'état $\ket{1} $, $\mathtt{CPS}\ket{11}
=e^{i\delta}\ket{11}$. La porte \texttt{Controlled Phase-Shift} n'a pas
d'analogue classique.

\begin{exercise}
 \begin{enumerate}
\item Find the action of the $\mathtt{CW}$ gate using (\ref{eq:CU}) when the
input state are $\ket{01}$ and $\ket{11}$.
\item Give the matrix representation and the Dirac notation representation of
$\mathtt{CW}$.
\end{enumerate}
\end{exercise}

\begin{footnotesize}
\begin{solution}
\begin{enumerate}
\item From (\ref{eq:CU}) is this easy to obtain
\begin{subequations}
\begin{align}
\mathtt{CW}\ket{01} & =\ket{0}W^0 \ket{1}=\ket{0}\ket{1}=\ket{01},\\
\mathtt{CW}\ket{11} & =\ket{1}W^1 \ket{1}=\ket{1}\frac{1}{\sqrt{2}}
(\ket{0}-\ket{1})= \frac{1}{\sqrt{2}}(\ket{10}-\ket{11}).
\end{align}
\end{subequations}
\item The matrix representation and the Dirac notation representation are
respectively
\begin{subequations}
\begin{align}
 \mathtt{CW} & =\begin{pmatrix}\mathbb{I} & \mathbb{O}\\
 \mathbb{O} & W\end{pmatrix}
=\begin{pmatrix} 1 & 0 & 0 & 0\\0 & 1 & 0 & 0\\
     0 & 0 & \frac{1}{\sqrt{2}} & \frac{1}{\sqrt{2}}\\
     0 & 0 & \frac{1}{\sqrt{2}} & -\frac{1}{\sqrt{2}}
    \end{pmatrix}\\
\mathtt{CW} &
=\ket{00}\bra{00}+\ket{01}\bra{01}+\frac{1}{\sqrt{2}}(\ket{10}\bra{10}+
\ket{10}\bra{11}+\ket{11}\bra{10}-\ket{11}\bra{11}).
\end{align}
\end{subequations}
\end{enumerate}
\end{solution}
\end{footnotesize}

\begin{remark}
La porte \texttt{CZ} ou \texttt{CMINUS} est définie par \texttt{CPS}$(\pi)$
=\texttt{CZ}, soit%

\begin{minipage}[c]{.5\linewidth}
 \[
\mathtt{CZ}=\begin{pmatrix}
1 & 0 & 0 & 0\\
0 & 1 & 0 & 0\\
0 & 0 & 1 & 0\\
0 & 0 & 0 & -1
\end{pmatrix}
\]
\end{minipage} \hfill\begin{minipage}[c]{.5\linewidth}
$\Qcircuit @C=1em @R=1.em {
\lstick{\ket{x}} & \qw  &  \ctrl{1} & \qw & \qw & \rstick{\ket{x}} \\
\lstick{\ket{y}} & \gate{W} &  \gate{X} & \gate{W} &\qw  &
\rstick{(-1)^{xy}\ket{y}} \\
&  & \mbox{Porte \texttt{CZ}} &
}$
\end{minipage}
\medskip

Cette porte est importante en ce sens qu'elle est plus facile à implémenter
que la porte \texttt{CX}. Comme $\mathtt{CZ}^{-1}=\mathtt{CZ}$, l'action de
$\mathtt{CZ}$ ne dépend pas de quel qubit est la cible ou le contrôle.

On a les relations suivantes entre \texttt{CX} et
\texttt{CZ}%
\begin{subequations}%
\begin{align}
\mathtt{CZ}  & =(\mathbb{I}\otimes W)\mathtt{CX}(\mathbb{I}\otimes W)\\
\mathtt{CX}  & =(\mathbb{I}\otimes W)\mathtt{CZ}(\mathbb{I}\otimes W)
\end{align}%
\end{subequations}%
\end{remark}

\begin{exercise}
 Montrer que l'équivalence suivante:
\[
 \Qcircuit @C=1em @R=1.em {
& \gate{W}  &  \ctrl{1} & \gate{W} & \qw &
\push{\rule{.3em}{0em}\equiv\rule{.3em}{0em}}  &&  \gate{X} &  \qw  \\
 & \gate{W} &  \gate{X} & \gate{W} &\qw  &&&  \ctrl{-1} &\qw
}
\]
\end{exercise}

\begin{remark}
Nous utiliserons une notation abrégée pour exprimer les opérateurs agissant sur
un ou plusieurs qubits d'un registre à $n$ qubits.
\begin{itemize}
\item Si $U$ agit sur $\mathcal{H}^{\otimes 2}$, alors
$U_{[i]}=\mathbb{I}^{\otimes i-1}\otimes U\otimes\mathbb{I}^{\otimes n-i}$.

\item Si $U$ agit sur $\mathcal{H}^{\otimes 2}\otimes\mathcal{H}^{\otimes 2}$
alors $U_{[i,j]}$ agit dans l'ordre sur les qubits $i$ et $j$ sans toucher les
autres qubits. Par exemple, $\texttt{CNOT}_{[2,1]}\ket{x}\ket{y}=\ket{x\oplus
y}\ket{y}$.
\end{itemize}

Cette définition peut se généraliser à un nombre arbitraire de qubits.
\end{remark}

\section{Portes quantiques universelles}
\label{sec:PQUniv}

L'intérêt de ces portes universelles est de faciliter l'intégration à partir
de portes pré-caractérisées.

Nous savons déjà que n'importe quelle fonction peut être synthétisée à l'aide
de:
\begin{itemize}
\item {\texttt{porte NAND, constantes 0 et 1}} dans le cas classique,

\item {\texttt{porte CNOT, portes single-qubit}}
($\mathtt{W},\mathtt{R}_{\boldsymbol{u}}(\delta)$) dans le cas quantique. On dit
que (\texttt{CNOT}, $W,R_{\boldsymbol{u}}(\delta)$) forme un ensemble
\emph{infini} de portes quantiques universelles.
\end{itemize}

Il existe cependant d'autres ensembles de portes quantiques universelles.

\subsection{Porte \texttt{CV}}

\medskip\colorbox[gray]{0.8}{
\parbox[c]{0.9\textwidth}{
\begin{proposition}
Toute porte quantique qui peut intriquer deux qubits peut être utilisée comme
porte quantique universelle. Mathématiquement, un choix élégant consiste en une
paire de portes de \texttt{Walsh-Hadamard} et des portes \texttt{CV}, où
$\mathtt{V}$ est la matrice%
\begin{equation}
\mathtt{V}=\begin{pmatrix}
1 & 0\\
0 & i
\end{pmatrix}
\equiv \mathtt{R}_{z}(\frac{\pi}{2}).
\end{equation}
\end{proposition}
}}\medskip

Ces deux portes forment un ensemble fini de portes quantiques universelles.
Les circuits quantiques contenant alors un nombre fini des portes $\mathtt{W}$
et \texttt{CV} peuvent implémenter toute transformation unitaire sur $n\geq2$
qubits.

Lorsqu'on applique quatre fois \texttt{CV}, on obtient l'identité, ainsi trois
applications consécutives de \texttt{CV} donne l'inverse de \texttt{CV} ou
\texttt{CV}$^{\dagger}$.

On construit une porte \texttt{CNOT} à partir des portes $\mathtt{W}$ et
\texttt{CV} de la manière suivante:
\[
\Qcircuit @C=.5em @R=1.em {
& \ctrl{1} & \qw & \push{\rule{.3em}{0em}\equiv\rule{.3em}{0em}} & & \qw & \qw &
\ctrl{1} & \qw &
\ctrl{1} & \qw & \qw & \qw \\
& \targ & \qw & & & \gate{W}& \qw &
\gate{V} & \qw & \gate{V} & \qw & \gate{W}& \qw
}
\]

On montre que toute matrice $2\times2$ unitaire $U$, telle que
$U=\mathtt{V}^2$, peut être simulée par le circuit
\[
\Qcircuit @C=.5em @R=1.em {
& \ctrl{1} & \qw & & & \qw & \ctrl{1} & \qw &
\ctrl{1} & \ctrl{2} & \qw\\
& \ctrl{1} & \qw &
\push{\rule{.3em}{0em}\equiv\rule{.3em}{0em}} & &
\ctrl{1} & \targ & \ctrl{1} & \targ & \qw &
\qw\\
& \gate{U} & \qw & & & \gate{V} & \qw &
\gate{V^\dag} & \qw & \gate{V} & \qw\\
& & & & & & \mbox{Construction de Sleator-Weinfurter}
}
\]

Il est noter que
\begin{equation}
\texttt{CCU}\ket{xyz}=\ket{xy}U^{xy}\ket{z}.
\end{equation}

\subsection{Porte de TOFFOLI}

Les portes $\mathtt{W}$ et \texttt{CV} permettent aussi de construire une porte
fort utile, à trois bits d'entrée et de sortie, appelée de \texttt{TOFFOLI} ou
porte \texttt{Controlled-Controlled-NOT (CCNOT, C$^2$NOT)}.
\[
\Qcircuit @C=.5em @R=1.em {
& \ctrl{1} & \qw & & & \qw & \qw & \ctrl{1} & \qw &
\ctrl{1} & \ctrl{2} & \qw & \qw &\qw \\
& \ctrl{1} & \qw &
\push{\rule{.3em}{0em}\equiv\rule{.3em}{0em}} & &
\qw & \ctrl{1}  &\targ & \ctrl{1} & \targ & \qw &
\qw & \qw &\qw \\
& \targ & \qw & & & \gate{W}& \gate{V} & \qw &
\gate{V^\dag} & \qw & \gate{V} & \qw & \gate{W}& \qw \\
& & & & &&&\mbox{Implémentation de la porte de \texttt{TOFFOLI}} &
}
\]

Cette porte, dont les deux bits de contrôle $x$ et $y$ restent inchangés alors
que le bit cible $z$ est inversé lorsque les deux bits de contrôle sont à $1$,
c'est-à-dire $z^{\prime}=z\oplus xy$, est représentée par la table de vérité
suivante:

\begin{minipage}[c]{.31\linewidth}
\[
\Qcircuit @C=1.em @R=1.4em {
\lstick{\ket{x}} & \ctrl{1} & \rstick{\ket{x}} \qw \\
\lstick{\ket{y}} & \ctrl{1} & \rstick{\ket{y}} \qw \\
\lstick{\ket{z}} & \targ & \rstick{\ket{z\oplus x\wedge y}} \qw \\
& \mbox{Porte de \texttt{TOFFOLI}}
}
\]
\end{minipage} \hfill\begin{minipage}[c]{.65\linewidth}
\centering
\begin{tabular}
[c]{|l|l|l|l||l|l|c|}\hline
\rowcolor[gray]{.8} N$^\circ$& $x$ & $y$ & $z$ & $x$ & $y$ & $z\oplus x\wedge
y$\\\hline
1&0 & 0 & 0 & 0 & 0 & 0\\\hline
2&0 & 0 & 1 & 0 & 0 & 1\\\hline
3&0 & 1 & 0 & 0 & 1 & 0\\\hline
4&0 & 1 & 1 & 0 & 1 & 1\\\hline
5&1 & 0 & 0 & 1 & 0 & 0\\\hline
6&1 & 0 & 1 & 1 & 0 & 1\\\hline
7&1 & 1 & 0 & 1 & 1 & 1\\\hline
8&1 & 1 & 1 & 1 & 1 & 0\\\hline
\end{tabular}
\end{minipage}
\medskip

La porte \texttt{CCNOT} nous donne la connectivité logique nécessaire à
l'arithmétique.
\begin{itemize}
\item Lorsque le qubit cible $\ket{z}$ est dans l'état $\ket{0}$ (lignes 1, 3,
5, 7), la porte de \texttt{CCNOT} effectue l'opération \texttt{AND}
\begin{equation}
\texttt{CCNOT}\ket{x}\ket{y}\ket{0}=\ket{x}\ket{y}\ket{x\wedge y}.
\end{equation}

\item Lorsque le qubit cible $\ket{z}$ est dans l'état $\ket{1}$ (lignes 2, 4,
6, 7), la porte de \texttt{CCNOT} effectue l'opération \texttt{NAND}
\begin{equation}
\texttt{CCNOT}\ket{x}\ket{y}\ket{1}=\ket{x}\ket{y}\ket{x\barwedge y}.
\end{equation}

\item Lorsque le premier qubit de contrôle $\ket{x}$ est dans l'état $\ket{1}$
(lignes 5-8), la porte de \texttt{CCNOT} effectue l'opération \texttt{CNOT}
\begin{equation}
\texttt{CCNOT}\ket{1}\ket{y}\ket{z}=\ket{1} \ket{y}\ket{z\oplus y}.
\end{equation}

\item Lorsque le premier qubit de contrôle $\ket{x}$ est dans l'état $\ket{1}$
et le qubit cible $\ket{z}$ est dans l'état $\ket{0}$ (lignes 5 et 7), la porte
de \texttt{CCNOT} effectue l'opération \texttt{COPY}
\begin{equation}
\texttt{CCNOT}\ket{1}\ket{y}\ket{0}=\ket{1}\ket{y}\ket{y}.
\end{equation}
\end{itemize}

Ainsi donc, avec la porte de \texttt{CCNOT}, on peut reproduire de façon
réversible tous les circuits logiques classiques.

\medskip\colorbox[gray]{0.8}{
\parbox[c]{0.9\textwidth}{
\emph{La porte de \texttt{CCNOT} est une porte universelle pour toutes
les opérations réversibles de la logique booléenne.}
}}

De ce qui précède, il apparaît que les portes logiques irréversibles comme
\texttt{AND} et \texttt{OR} peuvent être transformées en portes réversibles.
Cependant, le prix à payer est la production d'un bit ou de plusieurs
\textbf{bits résiduels} qui ne peuvent être re-utilisés pour le calcul. Ils
sont plutôt utiles pour le stockage de l'information à des fins de
réversibilité. Par exemple, lorsque dans la porte \texttt{CCNOT} $z=1$, on a
$z^{\prime}=x\wedge y$ et deux bits résiduels, $x^{\prime}=x$ et $y^{\prime
}=y$. On peut penser que de l'énergie est nécessaire pour effacer ces
résiduels et annuler l'effet avantageux de la réversibilité. Ce n'est point le
cas puisque d'après Bennett, on peut effectuer le calcul ou l'opération voulu,
imprimer le résultat et ensuite faire l'opération inverse, en utilisant à
nouveau les portes logiques, afin de retrouver l'état initial du calculateur.
Conséquemment, les bits résiduels reviennent à leur état initial sans dépense
d'énergie.

\begin{exercise}
Construct the \texttt{NOT} and \texttt{OR} gates from \texttt{CCNOT} gate.
\end{exercise}

\begin{exercise}
\textbf{Additionneur quantique.} Donner les expressions et les tables de vérité
de $S$ et $C$ du circuit suivant:
\[
\Qcircuit @C=1.em @R=1.4em {
\lstick{\ket{x}} & \ctrl{1} & \ctrl{1} & \rstick{\ket{x}} \qw \\
\lstick{\ket{y}} & \ctrl{1} & \targ & \rstick{\ket{S}} \qw \\
\lstick{\ket{0}} & \targ & \qw & \rstick{\ket{C}} \qw
}
\]
\end{exercise}

\subsection{Resumé}

On retient que

\medskip\colorbox[gray]{0.8}{
\parbox[c]{0.9\textwidth}{
\begin{itemize}
\item \emph{Pour toute rotation $U$ à un qubit, l'opération $\texttt{CU}$ peut
être décomposée en portes single-qubit et une porte \texttt{CNOT};}
\item \emph{La porte \texttt{CCNOT} peut être implémentée en utilisant les
portes \texttt{CNOT} et de \texttt{Walsh-Hadamard};}
\item \emph{Toute porte \texttt{C$^{k}$-U}, $k>2$, peut être décomposée
en portes de \texttt{CCNOT} et en portes \texttt{CU};}
\item \emph{Une opération unitaire générique $U^{(n)}$ agissant dans l'espace
$\mathcal{H}^{\otimes n}$ peut être décomposée au moyen des portes
\texttt{C}$^{k}$-U}.
\end{itemize}
}}

De façon réductrice, on peut dire qu'il suffit de savoir réaliser les rotations
single-qubits et des \texttt{CNOT} pour fabriquer le calculateur quantique!

Mais pourquoi donc est-ce si difficile? Parce que

\begin{itemize}
\item Il faut pouvoir agir sur chaque qubit séparément des autres;

\item Il faut pouvoir intriquer n'importe quelle paire de qubits;

\item Chaque opération doit être très précise;

\item Le calcul doit être effectué rapidement (\textbf{\emph{décohérence}}).
\end{itemize}

\subsection{Évaluation quantique d'une fonction}

Pour évaluer une fonction $f:\{0,1\}^n\rightarrow\{0,1\}^{m}$, on a besoin
d'au moins deux registres. Le premier, le registre de données (\emph{input
register}) de taille $n$ pour stocker les arguments $x$ de la fonction $f$, le
second, le registre de résultat (\emph{output register}) de taille $m$ pour
stocker les valeurs de $f(x)$. La fonction d'évaluation est une évolution
unitaire des deux registres,%
\begin{equation}
\ket{x,y} \longmapsto\ket{x,(y+f(x))\operatorname{mod}2^{m}},\ y\in\{0,1\}^{m}.
\end{equation}
Par exemple, le circuit calculant la fonction%
\begin{equation}%
\begin{array}
[c]{c}%
f:\{0,1\}^2\rightarrow\{0,1\}^{3}\\
x\rightarrow f(x)=x^2%
\end{array}
\end{equation}
agit ainsi qu'il suit%
\begin{equation}%
\begin{array}
[c]{ccc}%
\ket{00} \ket{000}\longmapsto\ket{00}\ket{000} &  & \ket{10}
\ket{000}\longmapsto\ket{01}\ket{100}\\
\ket{01} \ket{000}\longmapsto\ket{01}\ket{001} &  & \ket{11}
\ket{000}\longmapsto\ket{11}\ket{001}
\end{array}
\end{equation}
On peut l'écrire sous la forme%
\begin{equation}
\ket{x,0} \longmapsto\ket{x,x^2\operatorname{mod}2^{3}}.
\end{equation}
Comme $3^2\operatorname{mod}2^{3}=1$, on écrit $\ket{11}
\ket{000}\longmapsto\ket{11}\ket{001}$.

En réalité, pour ce genre d'opérations, on a aussi besoin d'un troisième
registre qui stocke initialement dans l'input des qubits à l'état zéro et
retourne des qubits à l'état zéro dans l'output, mais qui ont des valeurs
non-nulles durant le calcul.

Ce qui rend intéressant l'évaluation quantique d'une fonction est son action
sur la superposition d'état des différents inputs $x$. Par exemple,%
\begin{equation}
\sum_{x}\ket{x,0} \longmapsto\sum_{x}\ket{x,f(x)},
\end{equation}
produit $f(x)$ pour $x$ dans un cycle. Le bémol est que nous ne pouvons les
avoir toutes dans l'état intriqué $\sum_{x}\ket{x,f(x)} $ puisque tout bit
obtenu par mesure sur le premier registre donnera une valeur particulière
$x^{\prime}\in\{0,1\}^n$ et le deuxième registre se trouvera donc avec la
valeur $f(x^{\prime})\in\{0,1\}^{m}$.

\section{Algorithme de Deutsch-Jozsa}
\label{sec:DeutJ}

L'\textbf{algorithme de Deutsch-Jozsa} est un algorithme quantique, proposé
par David Deutsch et Richard Jozsa en 1992 avec des améliorations de R. Cleve
\emph{et al.} en 1998\footnote{R. Cleve, A. Ekert, C. Macchiavello, et M.
Mosca, \emph{\textbf{Quantum algorithms revisited}, Proceedings of the Royal
Society of London A}, vol. 454, 1998, p. 339-354}. \emph{Bien qu'il ne soit
pas d'un grand intérêt pratique, il s'agit d'un des premiers algorithmes
quantiques qui est plus efficace qu'un algorithme classique.}

\subsection{Problème et solution classique}

Dans le cas du problème de Deutsch-Jozsa, nous disposons d'une \emph{boîte
noire quantique}, connu sous le nom d'\textbf{oracle}, qui implémente une
fonction mathématique $f:\{0,1\}^n\rightarrow\{0,1\}$. Nous savons que cette
fonction est soit \emph{constante} ($f(0)=f(1)=0$ ou $f(0)=f(1)=1$) soit
\emph{équilibrée} ($f(0)=0,f(1)=1$ et $f(0)=1,f(1)=0$). Le but du problème est
de savoir si la fonction est constante ou équilibrée par invocation de l'oracle.

\begin{itemize}
\item Si un algorithme \emph{classique et déterministe} est utilisé, il faut
$2^{n-1}+1$ évaluation de la fonction mathématique $f$ dans le pire des cas
pour trouver la solution.

\item Dans le cas de l'utilisation d'un algorithme \emph{probabiliste}, un
nombre constant d'évaluation est suffisant pour trouver la bonne réponse avec
une forte probabilité, néanmoins $2^{n-1}+1$ évaluation sont toujours
nécessaire pour que la réponse soit toujours correcte.
\end{itemize}

L'algorithme quantique de Deutsch-Jozsa permet de trouver une réponse toujours
correcte avec une seule évaluation de $f$.


\subsection{Algorithme quantique de Deutsch-Jozsa}

Le but est de tester la parité de la fonction $f:\{0,1\}\rightarrow\{0,1\}$ ou
la condition $f(0)=f(1)$, ce qui équivaut à vérifier $f(0)\oplus f(1)$. Si
cela vaut zéro alors $f$ est constante, sinon $f$ est équilibrée. Le circuit
implémentant cet algorithme est donné par la figure \ref{fig:DJ1}.

\begin{figure}[htpb]
\[
\Qcircuit @C=1em @R=1em {
\lstick{\ket{0}} & \gate{W}& \ctrl{1} & \gate{W}& \meter \\
\lstick{\ket{1}} & \gate{W}& \gate{f} & \qw      & \qw &\qw
}
\]
\caption{Algorithme quantique de Deutsch-Jozsa avec un 1-qubit en entrée.}
\label{fig:DJ1} 
\end{figure}

\begin{enumerate}
\item L'algorithme commence avec deux qubit dans l'état $\ket{0}\ket{1}$.

\item Une transformation de Walsh-Hadamard est d'abord appliquée à chaque
qubit. Cela
donne
\begin{equation}
W\ket{0} W\ket{1}=\frac{1}{2}(\ket{0}+\ket{1})(\ket{0}-\ket{1}).
\end{equation}

\item Une implémentation quantique (oracle) de la fonction $f$ permet de
passer de $\ket{x}\ket{y}$ à $\ket{x}\ket{y\oplus f(x)}$, i.e., le
second qubit est inversé si et seulement si $f(x)=1$. Soit pour $x\in\{0,1\}$,%
\begin{equation}
U_{f}\ket{x}(\ket{0}-\ket{1})=(-1)^{f(x)}\ket{x}(\ket{0}-\ket{1}),
\end{equation}
où le facteur $(-1)^{f(x)}$ s'est retropropagé (\emph{kicked back})
devant le premier qubit. Ainsi, l'évaluation de la fonction nous donne%
\begin{equation}
\begin{split}
&  \frac{1}{2}\left[  (-1)^{f(0)}\ket{0}+(-1)^{f(1)}\ket{1}\right]
(\ket{0}-\ket{1}))\\
&  =(-1)^{f(0)}\frac{1}{2}(\ket{0}+(-1)^{f(0)\oplus f(1)}\ket{1}
)(\ket{0}-\ket{1}).
\end{split}
\end{equation}
Le second qubit n'est plus utile, de même que le facteur de phase global, on
peut donc les ignorer. On a alors l'état
\begin{equation}
\frac{1}{\sqrt{2}}(\ket{0}+(-1)^{f(0)\oplus f(1)}\ket{1}).
\end{equation}

\item En appliquant une transformation de Walsh-Hadamard à cet état, on a
\begin{equation}
\begin{split}
&  \frac{1}{2}(\ket{0}+\ket{1}+(-1)^{f(0)\oplus f(1)}\ket{0}
-(-1)^{f(0)\oplus f(1)}\ket{1})\\
&  =\frac{1}{2}([1+(-1)^{f(0)\oplus f(1)}])\ket{0}+[1-(-1)^{f(0)\oplus
f(1)}]\ket{1})  .
\end{split}
\end{equation}
$f(0)\oplus f(1)=0$, i.e., $f(0)=f(1)$ si et seulement si on observe
$\ket{0}$.

\item Donc, l'état final du premier qubit est $|f(0)\oplus f(1)\rangle$ et la
fonction est constante si et seulement si on mesure $\ket{0}$.
\end{enumerate}

Comme on le constate, la parité de la fonction $f(x)$ a été encodé par un
1-qubit après une seule invocation de $f$. Ceci parce qu'un calculateur
quantique peut évaluer simultanément $f(0)$ et $f(1)$. Les deux chemins
alternatifs ou complémentaires sont recombinés par la dernière porte de
\texttt{Walsh-Hadamard}. L'interférence est constructive pour l'une des valeurs
de $f(0)\oplus f(1)$ et destructive pour la valeur alternative.

Dans le cas général, on a $n+1$ bit dans l'état $\ket{0}^{\otimes n}\ket{1}$ et
la figure \ref{fig:DJn} représente l'implémentation de l'algorithme de
Deutsch-Jozsa correspondant.
\begin{figure}[htpb]
 \[
\Qcircuit @C=1em @R=1em {
\lstick{\ket{0}^{\otimes n}} & \gate{W^{\otimes n}}& \ctrl{1} &
\gate{W^{\otimes n}}& \meter \\
\lstick{\ket{1}} & \gate{W}& \gate{f} & \qw      & \qw &\qw
}
\]
\caption{Algorithme quantique de Deutsch-Jozsa avec un n-qubit entrée.}
\label{fig:DJn} 
\end{figure}

Les premiers $n$ bits sont tous dans l'état $\ket{0}$ et les derniers bit dans
l'état $\ket{1}$. Nous appliquons ensuite la transformation de Walsh-Hadamard à
chaque qubit, pour obtenir 
\begin{equation}
\ket{\psi_1} =\mathtt{W}^{\otimes n}\ket{0}^{\otimes n}\otimes\mathtt{W}
\ket{1} =\frac{1}{\sqrt{2^n}}\sum_{x\in\{0,1\}^{\otimes n}}\ket{x}
\left(\frac{\ket{0}-\ket{1}}{\sqrt{2}}\right). 
\end{equation}
Après l'oracle, le système est dans l'état
\begin{equation}
\ket{\psi_2} =\frac{1}{\sqrt{2^n}}\sum_x(-1)^{f(x)}\ket{x} \left(\frac{\ket{0}-
\ket{1}}{\sqrt{2}}\right).
\end{equation} 
Comme l'application de \texttt{W} sur le n-qubit $\ket{x}$ donne
\begin{equation}
 \mathtt{W}^{\otimes n}\ket{x}=\frac{1}{\sqrt{2^n}}\sum_y(-1)^{x\cdot y}\ket{y},
\end{equation} 
on a à la sortie du circuit
\begin{equation}
\ket{\psi_f} =\frac{1}{\sqrt{2^n}}\sum_x\sum_y(-1)^{x\cdot y+f(x)}\ket{y}
\left(\frac{\ket{0}-\ket{1}}{\sqrt{2}}\right).
\end{equation} 
Les résultats possibles de la mesure sur $\ket{y}$ donnent
\begin{itemize}
 \item seulement la valeur $0$ et dans ce cas la fonction $f(x)$ est constante,
 \item au moins une valeur $1$ et dans ce cas la fonction $f(x)$ est équilibrée.
\end{itemize}

\begin{exercise}
\begin{enumerate}
\item On considère une fonction à 2-qubit telle que $f(x)=1$. Montrer
explicitement que la sortie de l'algorithme de Deutsch-Jozsa donne
$\ket{y}=\ket{00}$.

\item On suppose que $f(00)=f(10)=0$ et $f(10)=f(11)=1$. Montrer
explicitement que des deux bits de la sortie $y$ de l'algorithme de
Deutsch-Jozsa au moins un à la valeur $1$.
\end{enumerate}

\end{exercise}

\section{Transformation de Fourier quantique}

La \textbf{transformation de Fourier \emph{discrète}} d'un vecteur avec $N$
composantes complexes $\{{f(0),f(1),\ldots,f(N-1)\}}$ est un nouveau vecteur
complexe $\{{\tilde{f}(0),\tilde{f}(1),\ldots,\tilde{f}(N-1)\}}$, défini par%
\begin{equation}
{\tilde{f}(y)=}\frac{1}{\sqrt{N}}\sum_{x=0}^{N-1}e^{2i\pi\frac{xy}{N}}f(x).
\label{eq:TFD}%
\end{equation}
La transformation de Fourier quantique (\textbf{QFT, Quantum Fourier Transform})
fait exactement la même chose. Elle est définie sur un registre de $n$ qubits
($N=2^n$) comme une opération unitaire $F_N$ dont l'action sur les états de la
base de calcul est%
\begin{subequations}%
\label{eq:QTFD}%
\begin{align}
\mathtt{F}_{N}\ket{x}=\frac{1}{\sqrt{N}}\sum_{y=0}^{N-1}e^{2i\pi\frac{xy}{N}}
\ket{y},
\end{align}%
ou
\begin{align}
\mathtt{F}_{N}=\frac{1}{\sqrt{N}}\sum_{x,y=0}^{N-1}\ket{y}e^{2i\pi\frac{xy}{N}}
\bra{x}.
\end{align}%
\end{subequations}%
Par conséquent, un état arbitraire $\ket{\psi}=\sum_{x}f(x)\ket{x}$ est
transformé en
\begin{equation}
\ket{\tilde{\psi}}=\sum_{k=0}^{N-1}{\tilde{f}(y)}\ket{y},
\end{equation}
où les coefficients $\{{\tilde{f}(y)}\}$ sont les transformées de Fourier
discrètes des coefficients $\{f(y)\}$, en vertu de (\ref{eq:TFD}).

On vérifie facilement que $\mathtt{F}_N$ est unitaire. En effet,%
\begin{subequations}%
\begin{align}
\mathtt{F}_N^{\dagger}F_{N}  &  =\left(  \frac{1}{\sqrt{N}}\sum_{y,x=0}%
^{N-1}\ket{x}e^{-2i\pi\frac{xy}{N}}\bra{y}\right)  \left(
\frac{1}{\sqrt{N}}\sum_{y,x=0}^{N-1}\ket{y} e^{2i\pi\frac{xy}{N}}\bra{x}
\right) \\
&  =\frac{1}{N}\sum_{y,x=0}^{N-1}\left(  \sum_{y,x=0}^{N-1}\ket{x} e^{-2\pi
i\frac{xy}{N}}\langle y\ket{y}e^{2i\pi\frac{xy}{N}}\bra{x} \right) \\
&  =\frac{1}{N}\sum_{x=0}^{N-1}\left(  \sum_{x=0}^{N-1}\ket{x}\bra{x}
\right)
=\mathbb{I}%
\end{align}%
\end{subequations}%

\begin{exercise}
Montrer que
\begin{equation}
\mathtt{F}_4=\frac{1}{2}\begin{pmatrix}
1 & 1 & 1 & 1\\
1 & i & -1 & -i\\
1 & -1 & 1 & -1\\
1 & -i & -1 & i
\end{pmatrix}.
\label{eq:F4}
\end{equation}
\end{exercise}

Si on pose $\omega_N=e^{\frac{2i\pi}{N}}$,
\begin{equation}
\mathtt{F}_N=\frac{1}{2}\begin{pmatrix}
1 & 1 & 1 & \cdots & 1\\
1 & \omega_N & \omega_N^2 & \cdots & \omega_N^{N-1}\\
1 & \omega_N^2 & \omega_N^4 & \cdots & \omega_N^{2(N-1)}\\
\vdots & \vdots & \vdots & \vdots & \vdots\\
1 & \omega_N^{N-1} & \omega_N^{2(N-1)} & \cdots & \omega_N^{(N-1)(N-1)}\\
\end{pmatrix}.
\end{equation}

Utilisons maintenant une technique standard des transformées de Fourier
rapides (\textbf{FFF} ou \textbf{Fast Fourier Transform}) pour montrer que la
QFT ne génère pas d'intrication entre les qubits sur lesquels elle agit.

Adoptons la représentation binaire pour écrire $x$ et $y$,
\begin{subequations}%
\begin{align}
x & =x_{n-1}x_{n-2}\cdots x_1x_0=x_{n-1}2^{n-1}+x_{n-2}2^{n-2}+
\cdots+x_12^1+x_02^0,\\
y & =y_{n-1}y_{n-2}\cdots y_1y_0=y_{n-1}2^{n-1}+y_{n-2}2^{n-2}+
\cdots+y_12^1+y_02^0.
\end{align}
\end{subequations}
En vertu de ce que $\omega_N=e^{\frac{2i\pi}{N}}$ est un racine N-ième de
l'unité, on peut ignorer dans le produit $xy$ de l'exponentielle de
(\ref{eq:QTFD}), tous les termes divisibles par $N=2^n$ puisqu'il ne contribuent
pas à l'exponentielle\footnote{Si $xy=aN+b$, avec $a,b\in\mathbb{N}$,
$\omega_N^{xy}=\omega_N^b$.}. Ainsi,
\begin{equation}
 \frac{xy}{2^n}=y_{n-1}(.x_0)+y_{n-2}(.x_1x_0)+\cdots+y_0(.x_{n-1}x_{n-2}\cdots
x_1x_0).
\end{equation}
Les termes entre parenthèses sont les fractions binaires, i.e.,
\begin{equation}
 0.x_{n-1}x_{n-2}\cdots x_1x_0=\frac{x_{n-1}}{2^1}+\frac{x_{n-2}}{2^2}+\cdots+
\frac{x_1}{2^{n-1}}+\frac{x_0}{2^n}.
\end{equation}
Ainsi pour, $y_i\in\{0,1\}$, l'amplitude
\begin{equation}
 \sum_{y=0}^{2^n-1}e^{2i\pi\frac{xy}{2^n}}=\left(\sum_ye^{2i\pi y_{n-1}(.x_0)}
\right)\cdots\left(\sum_ye^{2i\pi y_0(.x_{n-1}x_{n-2}\cdots x_1x_0)}\right),
\end{equation}
permet d'écrire l'équation (\ref{eq:QTFD}) sous la forme d'un produit tensoriel
\begin{equation}
\label{eq:QTFD2}
\mathtt{F}_N\ket{x}=\frac{1}{\sqrt{2^n}}[\ket{0}+e^{2i\pi(.x_0)}
\ket{1}]_{n-1}\otimes\cdots\otimes[\ket{0}+e^{2i\pi(.x_{n-1}x_{n-2}\cdots
x_1x_0)} \ket{1}]_0.
\end{equation}
La QFT ne génère donc pas un état intriqué.

\begin{example} Pour $n=2$
 \begin{equation}%
\begin{split}
\mathtt{F}_4\ket{x} & =\mathtt{F}_{4}\ket{x_1x_0}=\frac{1}{\sqrt{2^2}}[\ket{0}
+e^{2i\pi(.x_0)}\ket{1}]_1[\ket{0}+e^{2i\pi(.x_1x_0)}\ket{1}]_0 \\
& =\frac{1}{2}[\ket{00}+e^{2i\pi(.x_1x_0)}\ket{01}+e^{2i\pi(.x_0)}\ket{10}
+e^{2i\pi(.x_0+.x_1x_0)}\ket{11}].
\end{split}
\end{equation}
Si en plus on prend $\ket{x}=\ket{01}$,
\begin{equation}
\begin{split}
\mathtt{F_4}\ket{01}& =\frac{1}{2}[\ket{00}+e^{2i\pi(.01)}\ket{01}+e^{2i\pi(.1)}
\ket{10}+e^{2i\pi(.1+.01)}\ket{11}]\\
 & =\frac{1}{2}[\ket{00}+i\ket{01}-\ket{10}-i\ket{11}].
\end{split}
\end{equation}
Ce résultat est conforme à (\ref{eq:F4}).
\end{example}


\begin{figure}[ptbh]
\begin{footnotesize}
{\hspace*{2em}
$
\Qcircuit @C=1em @R=1.em {
\lstick{\ket{x_{n-1}}} & \gate{W}& \gate{R_2} & \qw & {\cdots} & &
\gate{R_{n-1}} & \gate{R_{n}} & \qw & \qw & \qw & \qw & \qw & \qw  & \qw & \qw &
\qw & \qw & \qw & \qw  & \qw &\qw & \rstick{\ket{y_{n-1}}} \\
\lstick{\ket{x_{n-2}}} & \qw & \ctrl{-1} & \qw & {\cdots}  & & \qw & \qw & \qw
&\gate{W}&\qw & {\cdots} & & \gate{R_{n-2}} & \gate{R_{n-1}} & \qw & \qw & \qw
& \qw & \qw & \qw  & \qw & \rstick{\ket{y}_{n-2}}\\
\lstick{\vdots} &\\
\lstick{\ket{x_1}}& \qw & \qw & \qw & \qw & \qw & \ctrl{-3} & \qw & \qw &
\qw & \qw \qw & \qw & \qw & \ctrl{-2} & \qw \qw & \qw & {\cdots} & &\gate{W}&
\gate{R_2} & \qw &\qw & \rstick{\ket{y}_1}\\
\lstick{\ket{x_0}}& \qw & \qw & \qw & \qw & \qw & \qw & \ctrl{-4} & \qw &
\qw & \qw & \qw & \qw  & \qw & \ctrl{-3} &\qw & {\cdots} & & \qw& \ctrl{-1}
&\gate{W}& \qw & \rstick{\ket{y}_0}
}
$
}
\end{footnotesize}
\caption{Circuit implémentant la transformation de Fourier quantique, avec
$\ket{y}_k=\frac{1}{\sqrt{2}}[\ket{0}+e^{2i\pi(.x_{k}x_{k-1}\cdots x_1x_0)}
\ket{1}]$. Les portes \texttt{SWAP} qui inversent l'ordre des qubits de sortie
ne sont pas représentées. On note que la transformation de Fourier d'un vecteur
complexe de taille $N=2^n$ peut être implémenté de manière efficiente sur un
registre de $n$ qubits en utilisant $n$ portes de \texttt{W} et
$\frac{n(n-1)}{2}$ portes $\mathtt{CR}_k$. Soit au total $\mathcal{O}(n^2)$
portes quantiques élémentaires. L'algorithme classique le plus efficace, la
transformée de Fourier rapide (FFF), calcule la transformée de Fourier discrète
en $\mathcal{O}(2^n n)$ opérations élémentaires.}
\label{fig:CircuitQFT}
\end{figure}

Le produit tensoriel (\ref{eq:QTFD2}) rend aisée l'implémentation d'un circuit
qui calcule la QFT de façon \emph{efficiente}. Un tel circuit, qu'illustre la
figure \ref{fig:CircuitQFT}, n'a besoin que des opérateurs de base $\mathtt{W}$
et \texttt{Controlled Phase Rotation} $\mathtt{CR}_{k\geq2}(\frac{2\pi}{2^{k}})$
avec
\begin{equation}
\mathtt{R}_{k}=\begin{pmatrix}
1 & 0\\
0 & e^{\frac{2i\pi}{2^{k}}}
\end{pmatrix}.
\end{equation}

Considérons l'action que circuit sur l'état $\ket{x_{n-1}x_{n-2}\cdots x_1x_0}$
de la base de calcul.

\begin{itemize}
\item La première porte de \texttt{W} agit sur le qubit le plus significatif
$\ket{x_{n-1}}$ et, en vertu de $e^{2i\pi(.x_{n-1})}=(-1)^{x_{n-1}}$, génère
l'état
\begin{equation}
(\mathtt{W}\ket{x_{n-1}})\ket{x_{n-2}\cdots x_1x_0}=\frac{1}{\sqrt{2}}
[\ket{0}+e^{2i\pi(.x_{n-1})}\ket{1}]_{n-1}\ket{x_{n-2}\cdots x_1x_0}.
\end{equation}

\item L'application de la porte $\mathtt{CR}_2$ génère l'état
\begin{equation}
\frac{1}{\sqrt{2}}[\ket{0}+e^{2i\pi(.x_{n-1})}e^{\frac{2i\pi x_{n-2}}{2^2}}
\ket{1}]_{n-1}\ket{x_{n-2}\cdots x_1x_0}=\frac{1}{\sqrt{2}}[\ket{0}
+e^{2i\pi(.x_{n-1}{x_{n-2}})}\ket{1}]_{n-1}\ket{x_{n-2}\cdots x_1x_0}
\end{equation}

\item Les portes $\mathtt{CR}_{k\geq3}$ subséquentes, $\mathtt{CR}_3$ à
$\mathtt{CR}_n$ ajoute les phases de $\frac{\pi}{2^2}$ à $\frac{\pi}{2^{n-1}}$
lorsque le qubit de contrôle à pour valeur $1$. Après ces $n-1$ portes à $2$
qubits, le calculateur quantique est dans l'état%
\begin{equation}
\frac{1}{\sqrt{2}}[\ket{0}+e^{2i\pi(.x_{n-1}x_{n-2}\cdots x_1x_0)}\ket{1}]_{n-1}
\ket{x_{n-2}\cdots x_1x_0}.
\end{equation}
\item La même procédure est répétée aux autres qubits et on obtient finalement%
\begin{equation}
\frac{1}{\sqrt{2^n}}[\ket{0}+e^{2i\pi(.x_{n-1}x_{n-2}\cdots x_1x_0)}\ket{1}
]_{n-1}\otimes\cdots\otimes[\ket{0}+e^{2i\pi(.x_0)}\ket{1}]_0.
\end{equation}
Cet état coïncide avec (\ref{eq:QTFD2}), excepté que le fait que l'ordre des
qubits est inversé. Le bon ordre est obtenu, soit en utilisant $\mathcal{O}(n)$
portes \texttt{SWAP}, soit en renumérotant simplement les qubits à la
sortie.
\end{itemize}

\begin{exercise}
Dessiner le circuit qui implémente la QFT du 3-qubit $\ket{x}=\ket{x_2x_1x_0}$
et évaluer pas à pas, l'état final de ce circuit.
\end{exercise}

\begin{remark}
Il faut souligner qu'on ne peut pas vraiment parler d'une accélération
exponentielle dans le calcul de la QFT, puisqu'un état arbitraire
$\ket{\psi}=\sum_{x}f(x)\ket{x}$ ne peut pas être préparé de manière efficace et
l'état transformé $\ket{\tilde{\psi}}=\sum_{k=0}^{N-1}{\tilde{f}(y)}\ket{y}$
n'est pas facilement accessible. En effet, une mesure standard donne simplement
un résultat avec $y$ probabilité $|\tilde{f}(y)|$. Le problème est que la
transformée de Fourier quantique est effectués sur les amplitudes de la fonction
d'onde, qui ne sont pas \emph{directement accessibles}. Ils ne peuvent être
reconstitués avec une précision finie qu'après de itérations (chaque itération
calcule la transformée de Fourier de l'état $\ket{\psi}$ et se termine avec une
mesure standard projective).
\end{remark}

\section{Réalisations physiques}

Les approches expérimentales proposées pour la réalisation d'un ordinateur
quantique sont nombreuses et variées:
\begin{itemize}
\item les qubits supraconducteurs à base de jonctions Josephson,
\item les qubits à boites quantiques semiconductrices,
\item les ions piégés dans le vide,
\item les spins nucléaires de molécules en solution, pilotés par résonance
magnétique nucléaire (RMN),
\item les atomes de Rydberg,
\item les défauts optiques cristallins, les cavités résonantes relevant du
domaine de l'optique quantique.
\end{itemize}

Dans un premier temps, il s'agit de réaliser une porte logique élémentaire,
c'est-à-dire d'arriver à coupler deux systèmes quantiques de façon suffisamment
forte pour que l'état de l'un puisse être modifié par l'état de l'autre avant
que la décohérence n'entre en jeu.

L'étape suivante consiste à augmenter le nombre de qubits en interaction pour
construire un vrai ordinateur. Des propositions ont été faites pour cette
dernière étape, mais la plupart soulève des difficultés technologiques
considérables.

Les points forts et faibles des différentes approches expérimentales sont
résumés dans le tableau \ref{tab:RealPhysExp}, qui reprend les cinq critères de
Di Vincenzo avec en plus, deux critères de connectivité:

\begin{itemize}
\item[\textbf{C1}]Capacité d'initialiser les qubits dans un état bien défini.

\item[\textbf{C2}]Capacité de mesurer les qubits avec fiabilité.

\item[\textbf{C3}]Temps de décohérence suffisamment long (beaucoup plus long que
le temps d'opération des portes quantiques).

\item[\textbf{C4}]Disposer d'un jeu universel de portes quantiques.

\item[\textbf{C5}]Capacité d'accommoder un grand nombre de qubits.

\item[\textbf{C6}]Capacité de convertir les qubits fixes en qubits mobiles.

\item[\textbf{C7}]Capacité de transmettre efficacement les qubits mobiles entre
plusieurs endroits.
\end{itemize}

\begin{table}[ptbh]
{
\newcommand{\mc}[3]{\multicolumn{#1}{#2}{#3}}
\begin{center}
\begin{tabular}{|l|l|l|l|l|l||l|l|}\hline
Approche Expérimentale & \mc{5}{l}{Critères de Di Vincenzo} &
\mc{2}{||l|}{Connectivité}\\\hline
 & C1 & C2 & C3 & C4 & C5 & C6 & C7\\\hline
Supraconducteur& $\spadesuit$ & $\spadesuit$ & $\clubsuit$ & $\clubsuit$ &
$\clubsuit$ & $\clubsuit$ & $\clubsuit$\\\hline
Semiconducteur & $\clubsuit$ & $\clubsuit$ & $\clubsuit$ & $\clubsuit$ &
$\clubsuit$ & $\blacktriangledown$ & $\blacktriangledown$\\\hline
RMN & $\clubsuit$ & $\clubsuit$ & $\clubsuit$ & $\spadesuit$ &
$\blacktriangledown$ & $\blacktriangledown$ & $\blacktriangledown$\\\hline
Ions piégés & $\spadesuit$ & $\spadesuit$ & $\clubsuit$ & $\spadesuit$ &
$\clubsuit$ & $\clubsuit$ & $\clubsuit$\\\hline
Cavités QED & $\clubsuit$ & $\spadesuit$ & $\clubsuit$ & $\spadesuit$ &
$\spadesuit$ & $\clubsuit$ & $\clubsuit$\\\hline
Atomes neutres & $\spadesuit$ & $\clubsuit$ & $\clubsuit$ & $\clubsuit$ &
$\clubsuit$ & $\clubsuit$ & $\clubsuit$\\\hline
Optique quantique & $\spadesuit$ & $\spadesuit$ & $\clubsuit$ & $\clubsuit$ &
$\clubsuit$ & $\clubsuit$ & $\clubsuit$\\\hline
\end{tabular}
\end{center}
}
\caption{\textbf{Synthèse des réalisations physiques}. $\spadesuit$=approche
viable ayant suffisamment fait ses preuves; $\clubsuit$=approche viable mais
requérant plus de preuves expérimentales; $\blacktriangledown$=aucune approche
viable connue actuellement. [Crédits: \emph{A quantum information science and
technology roadmap, report of the quantum information science and technology
experts panel}, v2.0 Avril 2004, \url{http://qist.lanl.gov/qcomp_map.shtml}]}
\label{tab:RealPhysExp}
\end{table}

\section{Conclusion}

L'ordinateur quantique a pu apparaître un temps comme un calculateur magique,
à l'origine de la révolution informatique du prochain millénaire. Ceci a été
largement sur-évalué par l'imagination populaire, à cause principalement du
mystère planant autour des concepts de théorie quantique, souvent mal compris
et mal interprétés. Il est très peu probable en effet, que l'ordinateur
quantique supplante intégralement l'ordinateur classique, pour la même raison
que la physique quantique ne remplace pas la physique classique pour la grande
majorité des problèmes quotidiens. Dans le monde macroscopique, personne ne
s'aventure en effet dans la résolution de l'équation de Schrödinger pour
concevoir un avion ou une voiture. De la même manière, l'ordinateur quantique
sera réservé à la résolution de problèmes particuliers, pour lesquels le
parallélisme quantique apporte un avantage décisif.

En particulier, l'ordinateur quantique donne l'espoir de pouvoir un jour
résoudre efficacement (en temps polynomial) des problèmes complexes, qui sont
insolubles avec des ordinateurs classiques, à cause de l'explosion exponentielle
de leur temps de calcul. Des algorithmes quantiques efficaces ont déjà vu le
jour, avec la découverte spectaculaire des \textbf{algorithmes de Shor et de
Grover} dans les années '90, suivie de la mise au point de nombreuses variantes
et améliorations. Malgré leur nombre encore restreint, ces algorithmes
quantiques sont théoriquement très efficaces, et apporteront à terme des
avantages calculatoires notables. S'ajoute à cela la mise au point de codes
correcteurs d'erreur, autorisant une certaine dose de décohérence sans laquelle
l'ordinateur quantique ne verrait certainement jamais le jour.

\newpage
\section{Exercices}


\subsection{Effets des erreurs d'amplitude et de phase}

On définit \emph{\textbf{l'erreur d'amplitude}} par la transformation%
\begin{equation}
\ket{\psi}\rightarrow\ket{\psi_{a}}=\beta\ket{0} +\alpha\ket{1} ,
\end{equation}
et \emph{\textbf{l'erreur de phase}} par la transformation%
\begin{equation}
\ket{\psi}\rightarrow\ket{\psi_{p}}=\alpha\ket{0} -\beta\ket{1} .
\end{equation}
Afin d'étudier les effets de ces deux transformations sur le qubit contrôle ou
cible d'une porte logique \texttt{CNOT}, on considère l'état initial%
\begin{equation}
(\alpha\ket{0} +\beta\ket{1})\otimes\frac{1}{\sqrt{2}}(\ket{0}+\ket{1}).
\end{equation}

\begin{enumerate}
\item Montrer que l'action de l'erreur de phase sur le qubit cible est
transférée au qubit de contrôle après application de l'opérateur porte logique
\texttt{CNOT}. On parle alors de \emph{propagation régressive du signe}.

\item Qu'en est-il lorsque l'erreur de phase agit d'abord sur le qubit de
contrôle et qu'on applique ensuite de l'opérateur porte logique \texttt{CNOT}?

\item Étudier de la même façon l'effet de l'erreur d'amplitude.
\end{enumerate}

\subsection{Opérateur racine carrée NOT}

On considère l'opérateur linéaire $\mathtt{H}=i\hbar\omega(\ket{0}\bra{1}
-\ket{1} \bra0)$ qui agit dans l'espace de Hilbert $\mathbb{C}^2$, où
$\{\ket{0},\ket{1}\}$ est une base orthonormée dans $\mathbb{C}^2$ et
$\omega\in\mathbb{R}$.

\begin{enumerate}
\item $\mathtt{H}$ est-il hermitien?

\item Quels sont les valeurs propres et les vecteurs propres normalisés de
$\mathtt{H}$?

\item Développer $\mathtt{U}(t)=e^{-i\mathtt{H}t/\hbar}$ sous la forme d'un
$\cos$ et d'un $\sin$ et trouver les valeurs de $t$ telles que $\mathtt{U}$
réalise l'opération \texttt{FNOT} (\textbf{Fake NOT}):%
\begin{equation}
\begin{cases}
\mathtt{U}(t)\ket{0} =-\ket{1} \\
\mathtt{U}(t)\ket{1} =\ket{0}
\end{cases}
\end{equation}

\item Calculer $\mathtt{U}(t=\frac{\pi}{4\omega})$ et 
$[\mathtt{U}(t=\frac{\pi}{4\omega})]^2$. Que peut-on conclure?

\item Sachant que traditionnellement en calcul quantique, l'opérateur
\texttt{NOT} est
\begin{equation}
	\mathtt{X}=\ket{0}\bra{1}+\ket{1}\bra0,
\end{equation}
utiliser l'opérateur $\mathtt{U}(t)=e^{i\frac{\pi}{2}}e^{-i\mathtt{H}t/\hbar}$,
$\mathtt{H}$ étant à définir, pour montrer que l'opérateur \textbf{racine carrée
NOT} est
\begin{equation}
	\mathtt{V}=\frac{1+i}{2}(\mathbb{I}_2-i\mathtt{X})=\frac{1+i}{\sqrt{2}}
	e^{-i\frac{\pi}{4}\mathtt{X}}.
\end{equation}
\end{enumerate}

\subsection{Algorithme quantique}

Soit $x_1,x_2,y_1,y_2\in\{0,1\}$ où Alice a $x_1$ et $x_2$ et Bob, $y_1$ et
$y_2$. Alice et Bob veulent calculer la fonction booléenne%
\begin{equation}
g(x_1,x_2,y_1,y_2)=x_2\oplus y_2\oplus(x_1\wedge y_1),
\end{equation}
où $\oplus$ désigne l'opération \texttt{XOR} et $\wedge$ l'opération
\texttt{AND}. En plus Alice et Bob échange la paire EPR%
\begin{equation}
\ket{\Phi^-}=\frac{1}{\sqrt{2}}(\ket{00} -\ket{11}).
\end{equation}
Alice applique la matrice unitaire $R(\theta_1)\otimes\mathbb{I}_2$ à son
qubit EPR alors que Bob applique au sien $\mathbb{I}_2\otimes R(\theta_2)$%
avec $\theta_1=-\frac{\pi}{16}+x_1\frac{\pi}{4}$,
$\theta_2=-\frac{\pi}{16}+y_1\frac{\pi}{4}$ et
\begin{equation}
\mathtt{R}(\theta)=\begin{pmatrix}
\cos\theta & -\sin\theta\\
\sin\theta & \cos\theta
\end{pmatrix}.
\end{equation}

On note $a$ le résultat de la mesure d'Alice sur son qubit EPR et $b$ le
résultat de la mesure de Bob sur le sien.

\begin{enumerate}
\item Donner l'expression de $\ket{\psi}=\mathtt{R}(\theta_1)\otimes R(\theta_2)
\ket{\Phi^-}$, l'état de la paire EPR après l'application de opérateurs de
rotation unitaires par Alice et Bob.

\item Si $\mathcal{P}(a,b)$ est la probabilité de trouver $a$ et $b$, remplir la
table de vérité suivante:
\begin{center}
\begin{tabular}
[c]{|c|c|c|c|}\hline
\rowcolor[gray]{.8}$a$ & $b$ & $a\oplus b$ & $\mathcal{P}(a,b)$\\\hline
$0$ & $0$ &  & \\\hline
$0$ & $1$ &  & \\\hline
$1$ & $0$ &  & \\\hline
$1$ & $1$ &  & \\\hline
\end{tabular}
\end{center}
\item Remplir la table de vérité suivante
\begin{center}
\begin{tabular}
[c]{|c|c|c|c|}\hline
\rowcolor[gray]{.8}$x_1$ & $y_1$ & $x_1\wedge y_1$ & $\mathcal{P}(a
\oplus b=x_1\wedge y_1)$\\\hline
$0$ & $0$ &  & \\\hline
$0$ & $1$ &  & \\\hline
$1$ & $0$ &  &\\\hline
$1$ & $1$ &  & \\\hline
\end{tabular}
\end{center}
 et en déduire la probabilité pour que%
\begin{equation}
a\oplus b=x_1\wedge y_1.
\end{equation}
\end{enumerate}


\subsection{Circuit intraportation}
\label{sec:CircIntr}

\emph{La \textbf{téléportation quantique} est un protocole de communications
quantiques consistant à \textbf{transférer} l'état quantique d'un système vers
un autre système similaire et séparé spatialement du premier en mettant à profit
l'intrication quantique.} Nous l'avons introduite à la section
\ref{sec:TelQ}. Résumons la problématique.

Il était une fois Alice et Bob qui, avant de se séparer, prirent chacun un qubit
d'une même paire EPR $\ket{\Phi^{+}}$. Puis Bob s'en alla, vers une galaxie
ignorée d'Alice.

C'est alors que, bien plus tard, un qubit dans un état inconnu, $\ket{\psi}
=\alpha\ket{0}+\beta\ket{1}$, arriva chez Alice. Mission d'Alice: transmettre
l'état $\ket{\psi}$ à Bob. Mais Alice ne pouvait pas
\begin{itemize}
\item porter ce qubit à Bob,
\item ni cloner $\ket{\psi}$ pour en disperser des copies dans l'univers,
\item ni connaître $\alpha$ et $\beta$ pour diffuser leurs valeurs sur les ondes
dans l'espace intergalactique.
\end{itemize}
Alors Alice devait téléporter $\ket{\psi}=\alpha\ket{0}+\beta\ket{1}$. Ils
avaient prévu tout ce qu'il fallait:
\begin{itemize}
 \item Un \emph{canal quantique}, la paire EPR $\ket{\Phi^{+}}$, dont ils
avaient soigneusement gardé chacun un qubit avant de se séparer.

\item Deux opérateurs, \texttt{W} et \texttt{CX}, qu'Alice avait chez elle, et
deux détecteurs $D_0,D_1$.

\item Et quatre autres opérateurs unitaires $U\equiv\mathbb{I},\mathtt{X},
i\mathtt{Y},\mathtt{Z}$ que Bob avait pris dans son vaisseau.
\end{itemize}

Le circuit quantique ci-dessous qui présente cette téléportation\footnote{Voir
G. Brassard, S.L. Braunstein and R. Cleve, \emph{Teleportation as a quantum
computation}, Physica \textbf{D120}, 43, (1998). Ce circuit est très souvent
appelé \textbf{intraportation} puisque la portes \texttt{CX} s'exécutent entre
les premier-second et second-troisième qubits. Ainsi, pour implémenter ces
portes \texttt{CX}, les deux premiers qubits ne peuvent être arbitrairement
éloignés du troisième.}. La première ligne représente le qubit $\ket{\psi}$ à
téléporter. La deuxième ligne appartient à Alice et la troisième ligne
appartient à Bob. La mesure effectuée par Alice (BSM), à travers les détecteurs
$D_0$ et $D_1$, donne deux bits classiques $b_0$ et $b_1$ (communiqués par un
canal classique (Téléphone, Internet) à Bob) qui conditionnent ou contrôlent la
transformation unitaire \texttt{U} effectuée par Bob.

\[
\Qcircuit @C=1em @R=1.em {
&&&&\mbox{BSM} \gategroup{2}{4}{3}{6}{.7em}{--}\\
\lstick{\ket{\psi}} &\qw &\qw
&\ctrl{1} &\gate{W} &\measureD{D_0} &\qw &\ctrl{2} & \rstick{b_0} \qw \\
\lstick{\ket{0}}    & \gate{W}&\ctrl{1}& \gate{X} &\qw & \measureD{D_1} &
\ctrl{1} & \qw    &\rstick{b_1} \qw \\
\lstick{\ket{0}}  & \qw &\gate{X}  &\qw &\qw  & \qw &\gate{X} & \gate{Z}
 &\rstick{\ket{\psi}} \qw \\
 & \mbox{\hspace*{2em}EPR}\gategroup{3}{2}{4}{3}{.7em}{--}&
&&&&\mbox{\hspace*{2em}U}\gategroup{2}{7}{4}{8}{.7em}{--}\\
&& & &\mbox{Circuit intraportation}
}
\]

\begin{enumerate}
 \item Donner l'expression de la paire EPR $\ket{\Phi^{+}}$ que Alice et Bob se
sont partagés.

\item Alice fait interagir $\ket{\psi}$ avec sa moitié de l'EPR. Exprimer le
3-qubit $\ket{\psi}_{123}=\ket{\psi}\otimes\ket{\Phi^{+}}$ qu'elle obtient dans
la base de Bell
$\{\ket{\Phi^{+}},\ket{\Phi^{-}},\ket{\psi^{+}},\ket{\psi^{-}}\}$, avec
$\ket{\Phi^{\pm}}=\frac{1}{\sqrt{2}}(\ket{00}\pm\ket{11})$,
$\ket{\psi^{\pm}}=\frac{1}{\sqrt{2}}(\ket{01}\pm\ket{10})$.

On posera
$\ket{\psi_0}=\alpha\ket{0}+\beta\ket{1}$,
$\ket{\psi_1}=\alpha\ket{1}+\beta\ket{0}$,
$\ket{\psi_{2}}=\alpha\ket{0}-\beta\ket{1}$,
$\ket{\psi_{3}}=\alpha\ket{1}-\beta\ket{0}$.

\item Lorsque Alice effectue des mesures de Bell (c'est-à-dire dans la
base de Bell), quelles résultats peut-elle obtenir et avec quelles probabilités?

\item Cependant, les états propres des appareils de mesure ou détecteurs d'Alice
ne sont pas les états de Bell, mais états standards $\ket{b_0b_1}\equiv
\{\ket{00},\ket{01},\ket{10},\ket{11}\}$.
\begin{enumerate}
\item Trouver la porte q-logique \texttt{B} qui permet le passage base de Bell
$\rightarrow$ base standard ($B\ket{\Phi^{+}}=\ket{00},B\ket{\Phi^{-}}
=\ket{10},B\ket{\psi^{+}}=\ket{01},B\ket{\psi^{-}}=\ket{11}$), en représentation
matricielle et comme produit des opérateurs 1-qubit et \texttt{CX}.

\item Quelle est maintenant l'expression de $\ket{\psi}_{123}$ (dans la base
standard)?
\end{enumerate}

\item Alice mesure les deux bits classiques en sa possession dans la base
standard. Cette mesure projette l'état du qubit de Bob dans un des états
$\ket{\psi_{i}}$, $i=0,1,2,3$. Faire un tableau où apparaîtra les résultats
possibles de la mesure d'Alice, l'état du qubit de Bob et les probabilités
correspondantes.

\begin{center}
\begin{tabular}
[c]{|p{4cm}|p{4cm}|p{3cm}|p{3cm}|}\hline\hline
\rowcolor[gray]{0.8}Résultat de la mesure de Alice &
État du qubit projeté reçu par Bob & Probabilité & \texttt{U}\\\hline\hline
&  &  & \\\hline
&  &  & \\\hline
&  &  & \\\hline
&  &  & \\\hline
\end{tabular}
\end{center}

\item Alice transmet à Bob par un canal classique le résultat de sa mesure, et
Bob sait que le qubit $\ket{\psi_{i}}$ lui arrive dans l'état inconnu de départ,
mais qui reste tout aussi inconnu!

En fonction de $\ket{\psi_{i}}$, dire quel opérateur unitaire \texttt{U} (que
l'on exprimera en fonction des puissances de $\mathtt{X},\mathtt{Z}$) Bob doit
appliquer pour reconstituer $\ket{\psi}$. Ce résultat sera inclut dans le
tableau de la question précédente.

\textbf{ L'état du qubit de départ $\ket{\psi}$ a été téléporté, mais il n'y a
jamais eu une mesure de cet état.}

\item Montrer, en parcourant le circuit, qu'on a effectivement à la sortie de
la 3e ligne, $\ket{\psi}$.

\end{enumerate}

\subsection{QuTiP - Circuit intraportation}
En utilisant le logiciel QuTiP, simuler le circuit de l'exercice 
\ref{sec:CircIntr}. Votre programme devrait faire ce qui suit:

\begin{enumerate}
\item Définir toutes les portes logiques du circuit intraportation.
\item Définir particulièrement l'opérateur \texttt{B} qui permet de générer 
la paire EPR $\ket{\Phi^{+}}$ partager par Alice et Bob.
\item Définir l'état d'entrée $\ket{\psi}_{in}=\ket{\psi}\otimes\ket{\Phi^{+}}$ 
du circuit.
\item Évaluer l'état de sortie $\ket{\psi}_{out}$ du circuit.
\item Évaluer $\rho_{Bob}$, en prenant la trace partielle de $\ket{\psi}_{out}$,
\item Calculer $\bra{\psi}\rho_{Bob}\ket{\psi}$. Vous devez trouver le résultat $1$.
\end{enumerate}


\subsection{Téléportation d'une paire EPR}\label{CircuiEPR}

Le circuit quantique ci-dessous qui présente la téléportation d'une paire 
EPR\footnote{voir V.N. Gorbachev and A.I. Trubilko, \emph{Quantum teleportation 
of EPR pair by three-particle entanglement}, J. Exp. Phys. \textbf{91}, 894 
(2000).}. Les premières portes génèrent l'état intriqué%
\begin{equation}
\alpha\ket{01} +\beta\ket{10} ,
\label{eq:EPRTelep}%
\end{equation}
et l'état GHZ (Greenberger, Horne et Zeilinger)%
\begin{equation}
\frac{1}{\sqrt{2}}(\ket{000}+\ket{111}).
\end{equation}

Montrer qu'à la sortie, l'état EPR (\ref{eq:EPRTelep}) est retrouvé aux
deux dernières lignes du circuit.
\[
\Qcircuit @C=1em @R=1.em {
&\mbox{EPR}\\
\lstick{\alpha\ket{0}+\beta\ket{1}} & \ctrl{1} & \qw      & \qw      & \qw
&  \qw     & \qw        & \qw      & \qw      & \qw      & \qw      & \qw
& \ctrl{3} & \gate{X}     & \qw \\
\lstick{\ket{1}} 		    & \gate{X}    & \qw      & \qw      &
\gate{W}
& \ctrl{1} & \gate{W}  & \qw      & \ctrl{2} & \qw      & \qw      & \qw       &
\qw      & \qw       & \qw \\
\lstick{\ket{0}} 		    & \gate{W}& \ctrl{1} & \ctrl{2} & \qw
& \gate{X}    & \ctrl{1}   & \qw      & \qw      & \qw      & \ctrl{2} & \qw
& \qw      & \qw       & \qw \\
\lstick{\ket{0}}    		    & \qw      & \gate{X}    & \qw      & \qw
& \qw      & \gate{X}      & \gate{W}& \gate{X}    & \gate{W}& \qw      &
\gate{X}    &
\gate{X}    & \qw       & \qw \\
\lstick{\ket{0}}    		    & \qw      & \qw      & \gate{X}    & \qw
& \qw      & \qw        & \qw      & \qw      & \qw      & \gate{X}    &
\ctrl{-1}
& \gate{W}& \ctrl{-4} & \qw \\
&&\mbox{{GHZ}}&&&&&\mbox{Téléportation d'une pair EPR}
\gategroup{2}{2}{3}{2}{.7em}{--}\gategroup{4}{2}{6}{4}{.7em}{--}
}
\]


\subsection{QuTiP - Téléportation d'une paire EPR}
Simuler le circuit de l'exercice \ref{CircuiEPR} en utilisant QuTiP. Votre programme devrait

\begin{enumerate}
\item Définir les portes logiques quantiques du circuit;

\item Définir l'état d'entrée du circuit $\ket{EPR}\otimes\ket{GHZ}$
(on prendra $|\alpha|^2=|\beta|^2=\frac{1}{2}$). Il faudra au préalable 
évaluer les états $\ket{EPR}$ et $\ket{GHZ}$;

\item Évaluer l'état de sortie du circuit, y extraire, à travers la trace 
partielle, l'état final de la dernière ligne;

\item Comparer avec l'état initial.

\end{enumerate}

\subsection{Transformée de Fourier Quantique}

Évaluer la sortie du circuit quantique de la figure \ref{fig:QFT3},
où $\mathtt{W}$ la porte 1-qubit de Walsh-Hadamard,
$\mathtt{CR}_k=\ket{0}\bra{0}\otimes\mathbb{I}+\ket{1}\bra{1}\otimes\mathtt{R}
_k$, avec $\mathtt{R}_k=\ket{0}\bra{0}+e^{\frac{2\pi i}{2^{k}}}\ket{1}\bra{1}$.

\begin{figure}[ptbh]
\[
\Qcircuit @C=1em @R=1.em {
\lstick{\ket{1}} & \gate{W} & \gate{R_2} & \gate{R_3} & \qw & \qw & \qw & \qw &
\qw & \qswap & \qw \\
\lstick{\ket{0}} & \qw & \ctrl{-1} & \qw & \gate{W} & \gate{R_2} & \qw & \qw  &
\qw & \qw \qwx &\qw \\
\lstick{\ket{1}} & \qw & \qw &\ctrl{-2} &\qw & \ctrl{-1} &\gate{W} & \qw  & \qw
 & \qswap \qwx & \qw }
\]
\caption{Circuit implémentant la Transformation de Fourier Quantique (QFT) de
$\ket{101}$.}
\label{fig:QFT3}
\end{figure}