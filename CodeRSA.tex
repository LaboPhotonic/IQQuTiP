\chapter{Code RSA}
\label{chap:RSA}
Créé en 1977 par Rivest, Shamir et Adelman, ce code de cryptographie, a bâti
sa fiabilité sur la difficulté de factoriser des grands nombres. Cependant, en
novembre 2005, au moyen de cinq mois de calculs complexes réalisés sur plus de
80 ordinateurs en réseau, des mathématiciens allemands réussissaient à
factoriser un nombre de 193 chiffres, remportant ainsi un défi lancé par RSA.
On pense qu'avec des ordinateurs quantiques effectuant des calculs parallèles,
on peut relever le défi en quelques secondes.

L'algorithme de ce code est le suivant:

\begin{enumerate}
\item Bob choisit deux nombres premiers $p$ et $q$ suffisamment grands et
calcule $N = pq$.

\item Bob choisit au hasard un nombre $c$ n'ayant pas de diviseur commun avec
le produit $(p-1)(q-1)$, c'est-à-dire $\gcd(c,(p-1)(q-1))=1$.

\item Bob calcule $d$ qui est l'inverse de $c$ pour la multiplication modulo
$(p-1)(q-1)$
\begin{equation}
cd\equiv1\pmod{(p-1)(q-1)}.
\end{equation}


\item Bob publie la paire $(d,N)$. C'est la clé publique que n'importe qui
peut utiliser pour envoyer un message à Bob.

\item La paire $(c,N)$ est la clé privée que possède seule Bob. Ainsi, il est
le seul à pouvoir déchiffrer un message chiffré au moyen de la clé publique.

\item Alice veut envoyer à Bob un message codé, qui doit être représenté par
un nombre $a<N$. Si le message est trop long, Alice le segmente en plusieurs
sous messages $a_i<N$. Elle chiffre ensuite chaque sous-message en
calculant,
\begin{equation}
b\equiv a_i^{d}\pmod{N},
\end{equation}
et envoie $b$ à Bob.

\item Quand Bob reçoit le message qu'il déchiffre en calculant
\begin{equation}
b^{c}\pmod{N}=a_i(!)
\end{equation}


En effet, le fait que le résultat soit précisément $a_i$, c'est-à-dire le
message original d'Alice, est un résultat de théorie des nombres.
\end{enumerate}

En résumé, sont envoyés sur voie publique, non sécurisée, les nombres $N$, $d$
et $b$. Les avantages par rapport au code \emph{one-time Pad} sont:

\begin{itemize}
\item Il n'est point besoin de distribuer une clé secrète par un canal supposé
sécurisé. La clé publique peut être utilisée par quiconque veut communiquer
avec Bob, lequel possède seul la clé sécrète.

\item La clé publique peut être re-utilisée autant de fois qu'on le désire.
\end{itemize}

\begin{example}
$p=3$,\ $q=$, $7N=21$, $(p-1)(q-1)=12$.

$c=5$ n'a aucun facteur commun avec $12$, et son inverse par rapport à la
multiplication modulo $12$ est $d=5$ car $5\times5=24+1$. Alice choisit pour
message $a=4$. Elle calcule

\begin{equation}
4^{5}=1024=21\times48+16,\qquad4^{5}=16\pmod{21}
\end{equation}
Alice envoie donc à Bob le message $16$. Bob calcule

\begin{equation}
b^{5}=16^{5}=49\ 932\times21+4,\qquad16^{5}=4\pmod{21}
\end{equation}
et Bob récupère donc le message original $a=4$.
\end{example}

